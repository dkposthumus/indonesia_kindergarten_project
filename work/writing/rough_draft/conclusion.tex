In this research, I examine the effects of kindergarten on schooling and cognitive performance. Using data from the Indonesian Family Life Survey (IFLS) and Village Potential Statistics (PODES), I examine the effects of kindergarten on educational outcomes in Indonesia, focusing on schooling and cognitive performance. My empirical strategy entails ordinary least-squares (OLS), mother fixed-effects, and instrumental variable (IV) estimation, where my instrument is the number of kindergartens per 10,000 individuals in each locality.

Broadly, my results suggest that kindergarten has a positive relationship with educational outcomes, particularly for schooling. Kindergarten is associated with children completing more years of education and there being a greater likelihood of children attending school in junior high school. This is an important finding and is promising for policymakers in Indonesia who want to continue expanding school attendance beyond primary school.

My results also provide evidence for two interlinked phenomena critical to understanding human capital formulation in Indonesia: 1) `fadeout' in the effect of kindergarten and 2) divergence between the effects of kindergarten on schooling and learning. First, there is considerable heterogeneity in kindergarten effects over time: in short, kindergarten has a greater impact on short- and medium- term outcomes rather than on long-term ones. I analyze this by breaking down schooling into school completion, school attendance, and the decision to stay-on after the completion of primary, junior high, and senior high school. 

Disregarding the elementary school years, for which there is very little variation in outcomes, my results provided robust evidence that kindergarten increased the likelihood a student would attend school post-elementary school and complete junior high. There was also mixed evidence supporting the hypothesis that kindergarten had a positive effect on staying on after the completion of junior high school, and the completion of senior high school. Turning to cognitive performance, my OLS estimates found significant presence of fadeout with significant positive effects of kindergarten on cognitive performance in 2000 and 2007, and an insignificant effect in 2014, while IV estimation found insignificant effects for all three years. These results cumulatively paint a picture where the effects of kindergarten are positive--until approximately junior high school, after which they are insignificant. Figure ~\ref{fig:inschl_reg} is an effective visual demonstration of fadeout. These results confirm the extensive evidence of `fadeout' that previous scholars have found, which are discussed in greater detail in Section ~\ref{sec:lit_rev}. 

To the second phenomenon, my results regarding the effect of kindergarten on cognitive performance are mixed. I regard the IV results, valid and robust for all three models involving cognitive performance, as more valid than the OLS results that suffer from omitted variable bias. The IV results suggest absolutely no significant effect for any of the 2000, 2007, or 2014 cognitive test performances. In fact, the fixed-effects model finds that kindergarten has a negative effect on the 2014 results, at a 90\% confidence level--one of the few significant findings of the fixed-effects model. This is finding parallels major concerns in the Indonesian education system, as there is a clear divergence between schooling and learning occurring--echoing the fears discussed in Section ~\ref{sec:intro} that increasing enrollment numbers in Indonesia were not translating to greater learning and thus not alleviating concerns about human capital formation. 

These results cumulatively suggest that there needs to be a closer examination of the quality and accessibility of kindergarten programs before early childhood education is expanded as a policy to improve human capital formation and, consequently, productivity. That data on kindergarten is unfortunately difficult to capture, the specifics of kindergarten only being asked about in the most recent survey wave of the IFLS, in 2014; however, with a new wave of the IFLS on the horizon, it would be possible to incorporate the 2014 data on kindergarten quality into an analysis on short- and perhaps medium-term outcomes.

There are a great number of ways that this research could be extended and improved. One would be to expand the number and types of outcome variables examined; all of the individuals in my sample were interviewed individually in 2014, so without further attrition I would be able to access variables on labor markets, health, and social outcomes.\footnote{Although due to the informal nature of the Indonesian economy, I would suspect that labor market outcomes harbor measurement error.} I could also examine intermediate outcomes; some of my sample were also interviewed in 2000 and 2007, and for those that left school before completing senior high school I could investigate their reasons for doing so, which are asked about in the individual interviews. Additionally, with the new wave of the IFLS being fielded soon (see above), I would be able to incorporate even longer-term outcomes, including--perhaps most promisingly--the role that kindergarten plays in intergenerational economic mobility. The difficulty of these extensions is the attrition of adding another wave of the IFLS to the survey, entailing a trade-off in sample size and sample detail. 