Indonesia is a rapidly developing economy: between 2000 and 2019, it averaged 5.26\% economic growth and its post-COVID \textit{economic} recovery is robust. Yet, human capital formation has lagged behind the country's impressive economic performance, resulting in slowing productivity growth. The pandemic hit the Indonesian education system particularly hard: starting in March 2022, Indonesian schools were closed for 21 months. Consequently, students lost approximately 11 months of learning--and poor children were even more negatively affected, widening inequality in learning outcomes \citep{Bank2023}. As a result of these weaknesses in the educational system, 2022 Programme for International Student Assessment (PISA) scores were lower across the board than 2018 scores; in 2022, Indonesia ranked near the bottom in nearly every indicator among the 80 countries that completed PISA tests \citep{Wijaya2024}.\footnote{PISA tests are fielded by the OECD and focus on 15-year-old academic performance. Declining test scores in Indonesia predate COVID, as Indonesia saw declines in PISA mathematics scores between 2015 and 2018 as well.} These declines are occurring amidst increases in education spending, student enrollment, and gender parity; clearly, there is a divergence between going to school and learning in school \citep{Afkar2020}.

One solution to this trend of worsening or stagnating educational outcomes might be expanded early childhood interventions: a World Bank report in 2020 proposed compulsory and accessible ``two years of quality early childhood education" as one way to address Indonesia's challenges in learning and skills attainment \citep{Afkar2020}. Currently, such interventions are sparse; in this project, I focus on \textit{formal} classroom early childhood programs, referred to as kindergarten. In Indonesia, kindergarten is not compulsory and--the same as in the United States--it is typically completed the year immediately prior to students beginning primary school. Currently and historically, the overwhelming majority of kindergartens are private, suggesting a lack of investment in accessible kindergarten.\footnote{See Figure ~\ref{fig:kinder_numscatter}. The current evidence comes from the 2020 World Bank Report referenced above: in 2019-2020, 95.7\% of kindergartens were private. In 1990 and 2000 that figure was 95.06\% and 96.63\%, respectively.} 

Early childhood education is a critical part of the process of human capital accumulation; there is a great deal of evidence that in early childhood, when brains are at their most malleable, human capital investments exhibit higher rates of return than equivalent investments later in life \citep{Cantor2019}. Certainly, there is empirical evidence supporting the efficacy of early childhood interventions in improving human capital and later-life outcomes, particularly for \textit{disadvantaged} children \citep{Duncan2023}.

This invites the question of whether kindergarten has previously been effective in improving educational outcomes in Indonesia--an under-studied question that is critical to informing future educational policy. Thus, in this research I examine the effects of kindergarten attendance, with an emphasis on looking at short-, medium-, and long-term effects. I focus on 1) schooling and 2) cognitive performance. I define schooling for the purpose of this paper narrowly as \textit{only} the \textit{completion} of schooling. I conceive of cognitive performance as a proxy for learning. Thus, one can think of the distinction between schooling and cognitive performance as the distinction between a child sitting in a classroom--completing schooling--and a child learning while sitting in the classroom--improving cognitive ability and building skills. To examine how the effects of kindergarten may vary over time, a critical concept in the early childhood intervention literature, I have outcome variables that vary over time, such as elementary school completion vs. junior high school completion, and cognitive tests taken during three different years. 

To study long-term effects, it is necessary to have a detailed multi-wave household survey in order to track individuals from pre-early childhood intervention to adulthood. To this end, I employ the Indonesian Family Life Survey (IFLS) to track individuals from 1997 to 2014, while also collecting household- and community-level data. The IFLS is richly detailed, and I use four waves of the survey—from 1997 (approximately when children in my sample are attending kindergarten), 2000 (when they are in elementary school), 2007 (when they are approximately in junior high school), and 2014 (when they are adults who are either in college or have left the educational system). I also use the Village Potential Statistics (PODES) survey to gather data on community-level presence of kindergartens. Merging across waves of the IFLS and narrowing my sample down to individuals who were between 3 and 9 years old in 1997, my sample consists of 3,158 individuals.

I use three empirical methods: ordinary least squares (OLS) estimation, mother fixed-effects (FE), and instrumental variable (IV) estimation. While my OLS results are strongly statistically significant and hold across various robustness checks, my results suffer from omitted variable bias, meaning that I am missing necessary explanatory variables in the model. This bias motivates my use of mother fixed-effects, which allows me to control for all unobserved characteristics of siblings' shared mothers; however, mother fixed-effects places additional restrictions on my sample which results in the loss of a significant portion of the sample. Thus, I supplement these two methods with instrumental variable (IV) estimation, using the number of kindergartens per 10,000 people in each locality in 1990 and 2000 as my two instruments. I find that intuitively and empirically these instruments are strong and valid, so long as I control for the number of primary/junior high/senior high schools in each locality.

I find strong evidence that kindergarten has a positive association with schooling, although that effect decreases in magnitude and significance as time passes. Specifically, I find significant positive results for kindergarten’s effects on total years of education, completion of elementary and junior high schools, school attendance in junior high, and the likelihood of a student continuing after the completion of elementary school. Critically, I find that kindergarten has little to no impact on schooling after junior high school, and no impact on cognitive test performance for any of the three years that individuals completed the test. 

The results suggest two phenomena: 1) `fadeout', the concept that the effects of early childhood intervention fade as time passes and 2) a divergence between completed schooling and learning. These findings provide evidence of the fears outlined above, that while schooling is rapidly improving in Indonesia, there is not an accompanying improvement in learning. Thus, before expanding preschool to improve human capital formation, my results suggest there needs to be more research into the \textit{quality} of existing kindergarten programs. Further research may focus on the effects of kindergarten on earnings, economic mobility, and adult health outcomes.

I begin this paper with a review of the relevant literature and how my research is situated in previous scholarly work on human capital accumulation and early childhood interventions. I then describe my sample and the origins of the data used for this research. Next, I introduce the three empirical methods I have used—OLS, mother fixed-effects, and IV estimation. I then present the results of my analysis, before concluding with a discussion of the interpretations of their results, their significance, and future research related to my question.