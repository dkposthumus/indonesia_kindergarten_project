\documentclass[11pt,final]{article}
\usepackage[paperwidth=8.5in,left=1.0in,right=1.0in,top=1.0in,bottom=1.0in,paperheight=11.0in]{geometry}
\usepackage{enumitem}
\usepackage[breaklinks]{hyperref}
\hypersetup{pdfdisplaydoctitle=true,bookmarksnumbered=true,colorlinks=true,citecolor=black,linkcolor=blue,urlcolor=red,pdfstartview=FitH,pdfpagemode=UseNone}
\usepackage[sort&compress]{natbib}
\usepackage[nottoc]{tocbibind}

\usepackage[nodisplayskipstretch]{setspace}
\usepackage[page,toc,titletoc,title]{appendix}
\doublespacing
\usepackage{times}
\usepackage{amssymb,amsmath,amsthm}
\usepackage{mathtools}
\usepackage{graphicx}
\usepackage{float}
\usepackage[labelsep=colon,small,singlelinecheck=off,labelfont=bf,font=large,justification=centering]{caption}
\usepackage[font+=small]{subcaption}
\usepackage{booktabs}
\usepackage{comment}
\usepackage{blindtext}
\usepackage{titlesec}
\usepackage[flushleft]{threeparttable}
	\newcommand{\figlist}[1]{\autoref{#1}: \nameref{#1}}
\makeatletter
\def\input@path{{tabs/}}
\makeatother

\newcommand{\writing}[1]{/Users/danielposthumus/thesis_independent_study/work/writing/rough_draft/#1}

\pagenumbering{roman}

\title{Starting Early: Returns on Kindergarten Attendance in Indonesia}
\author{Daniel Posthumus}
\date{May 2024}

\begin{document}
\newgeometry{left=2in,right=2in,top=1.25in,bottom=1.25in}
\begin{center}
{\Large Starting Early: Returns on Kindergarten Attendance in Indonesia} \\
\vspace{0.4in}
A thesis presented in Candidacy for Departmental Honors in Economics from The College of William and Mary \\
\vspace{0.2in}
by \\
\vspace{0.1in} 
Daniel Posthumus \\
\vspace{1in}
Accepted for \hrulefill \\
\vspace{0.3in}
\qquad \qquad \qquad \hrulefill \\
\vspace{-0.1in}
\end{center}
\qquad \qquad \qquad Professor Ranjan Shrestha \\
\begin{center}
\vspace{-0.25in}
\qquad \qquad \qquad \hrulefill \\
\vspace{-0.1in}
\end{center}
\qquad \qquad \qquad Professor John Parman \\
\begin{center}
\vspace{-0.25in}
\qquad \qquad \qquad \hrulefill \\
\vspace{-0.1in}
\end{center}
\qquad \qquad \qquad Professor Dan Cristol \\
\begin{center}
Williamsburg, VA \\
May 02, 2024
\end{center}
\thispagestyle{empty}
\clearpage
\restoregeometry

\maketitle
\thispagestyle{empty}

\begin{center}
\textbf{\large Abstract}
\end{center}
Indonesia is a rapidly developing economy, having averaged 5.26\% economic growth from 2000 to 2019; over the same time, it has achieved near-universal primary school attendance. However, there are concerns about the quality of Indonesian education, with no improvement in standardized test scores between 2012 and 2022. Early childhood interventions are a critical part of human capital accumulation and skills-building, and the efficacy of interventions such as kindergarten in developing countries like Indonesia is under-studied. Using data from the Indonesian Family Life Survey (IFLS) and Village Potential Statistics (PODES), I examine the effects of kindergarten on educational outcomes in Indonesia, focusing on schooling and cognitive performance. My empirical strategy entails ordinary least-squares (OLS), mother fixed-effects (FE), and instrumental variable (IV) estimation, where my instrument is the number of kindergartens per 10,000 individuals in each locality. I find that kindergarten has a significant positive association with schooling, associated with an additional 1.89 years of education. Additionally, I find evidence that kindergarten's positive association with educational outcomes fades out as time passes, there being little to no evidence of significant positive effects after the conclusion of junior high school. I also find little to no evidence that attending kindergarten has a significant association with performance on cognitive tests--suggesting there is a gap between schooling and skills learned in the classroom. My results motivate a closer look at this gap, as well as exploring the effects of kindergarten attendance on earnings or social outcomes such as delinquency. \\ \\
\textbf{Keywords:} early childhood education, human capital, kindergarten, development, Indonesia

\clearpage
	\tableofcontents
\clearpage
\noindent \textbf{\huge List of Figures} \\
\figlist{fig:kinder_binscatter} \\
\figlist{fig:kinder_numscatter} \\
\figlist{fig:num_schl} \\
\figlist{fig:kinder_prov} \\
\figlist{fig:scatter_switching} \\
\figlist{fig:inschl_reg} \\
\figlist{fig:ek_shapes} \\
\figlist{fig:ek_math}

\vspace{1.1in}

\noindent \textbf{\huge List of Tables} \\
\figlist{table:summary_table}\\
\figlist{table:attrition_table}\\
\figlist{table:iv_first_stage}\\
\figlist{table:full_results}\\
\figlist{table:iv_main_results}\\
\figlist{table:full_complete}\\
\figlist{table:full_stayon}\\
\figlist{table:full_ek}\\
\figlist{table:switching_stats}\\
\figlist{table:kinder_logit}\\
\clearpage

\noindent \textbf{\huge Acknowledgements} \\
I would like to express my deep gratitude to my wonderful advisor, Professor Ranjan Shrestha for his mentorship and for taking me to Indonesia this past summer. He has been teaching me about economics since ECON 101, which I took on Zoom my very first semester of college from my childhood bedroom in Japan. I could not have accomplished this without his immense wisdom, his everlasting patience, his generosity, and his commitment to pushing me. He is truly the best advisor and mentor one could ask for, and I am sure he will continue to teach me about economics for the rest of my life.

I also want to thank Professor Daniel Cristol and Kim Van Deusen of the 1693 Scholars Program, and Professor Cristol for serving on my Examination Committee. They have been an endless source of support for the past four years, providing amazing mentorship. I would also like to thank James Murray for his support of the 1693 Program and his tremendous financial support for my academic career. It is thanks to Professor Cristol, Ms. Van Deusen, and Mr. Murray that I was able to travel to Indonesia this past summer.

I extend my deepest appreciation to Sudarno Sumarto and Elan Satriawan of TNP2K, who welcomed me to Jakarta, Indonesia with the greatest of hospitality and taught me so much about economics and Indonesia--without the wisdom and experience they imparted upon me, I would not have been able to conduct this project. 

I also want to thank Professor John Parman for his serving on my Examination Committee, his pointed questions and detailed feedback greatly improved my thesis from defense to submission. My first, and longest, undergraduate research experience has been with Dr. Kelebogile Zvobgo in the International Justice Lab; I want to thank her for four years of mentorship and guidance, and helping show me how to be an effective researcher. I also want to thank the Charles Center for their financial support of my project.

Lastly, I could not have done this without my parents. As long as I can remember, they have pushed me to be more curious about the world and exposed to me new things--new places, new perspectives, new cultures, and new information--and I would not be who I am today without them.

\clearpage

\pagenumbering{arabic}
\section{Introduction}
	\label{sec:intro}
\input{\writing{introduction.tex}}

\section{Literature Review}
	\label{sec:lit_rev}
\input{\writing{lit_review.tex}}

\section{Data}
	\label{sec:data}
\def \data #1{/Users/danielposthumus/thesis_independent_study/work/writing/rough_draft/data/#1}

In this section, I describe the data and sample I use for my research. I begin by discussing my two sources of data--the Indonesian Family Life Survey (IFLS) and Village Potential Statistics (PODES) before I proceed to describing my sample and its construction.

\subsection{Indonesian Family Life Survey}
	\label{sec:ifls}
My main source of data is the Indonesian Family Life Survey (IFLS), a longitudinal household and community survey fielded through five waves from 1993 to 2014 by the RAND corporation. I use it for the entirety of my individual- and household-level data, as well as for the majority of my community-level data. The availability of the IFLS--a free, public dataset--makes Indonesia relatively unique among developing countries for having a detailed longitudinal household survey, which is necessary to study the medium- and long-term effects of early childhood interventions.

The original wave of the IFLS, fielded in 1993 and 1994, consists of 7,200 households and is representative of about 83\% of Indonesia's population at the time \citep{Serrato1995}. In each of the successive waves, the survey has attempted to re-interview the same households (with high rates of success in recontacting across the years), in order to create panel data. In the fifth wave of the IFLS, fielded in 2014, 16,204 households and 50,148 individuals were interviewed \citep{Strauss2016}. I discuss how I merged data from the survey's different waves and ensuing attrition in greater detail in Appendix ~\ref{app:attrition}.

I use the IFLS to track children from when they were young in 1997 to when they are adults in 2014. I focus on data from 1997 because that is the most recent wave of the IFLS in which kindergarten-age respondents will be at least 18 years old in the most up-to-date survey wave, fielded in 2014. Some of my covariates take the form of panel data--namely in the repeated observations of household-level variables such as household per capita expenditure and  size of household, or community-level variables such as the number of elementary schools per 10,000 people in each kecamatan, i.e., subdistrict.\footnote{For example, my independent variable--a dummy variable of whether an individual attended kindergarten--does not require repeated observation, nor do the \textit{outcomes} I am interested in, i.e., school/grade completion or years of education completed.}

\subsection{Village Potential Statistics}
	\label{sec:podes}
As comprehensive as the IFLS is, its community survey only began to inquire after the presence of kindergartens in a community in 2014: therefore, in order to construct my instrument--the number of kindergartens per 10,000 individuals in each kecamatan, I needed to supplement the IFLS with another data source.

The Village Potential Statistics (PODES) dataset is immensely powerful; PODES data is published every year, dating back to 1983. It is fielded by Indonesia's Office of Statistics (BPS) and is available only by purchase. As the price of purchasing PODES is dependent on the number of variables taken from the survey, I limit my use to the two variables I need: the presence of kindergartens and village population. I also limit the data to only the years I particularly need--1990 and 2000. My research focuses on children who were kindergarten-aged during the 1990s; by collecting data that bookend the period of interest, I attempt to maximize data coverage.

PODES is comprehensive and representative of Indonesia in its scope, with data from approximately 65,000 villages. Therefore, I sum the number of kindergartens (public, private, and total) and population by kecamatan and year to find the number of kindergartens per 10,000 individuals in each kecamatan for 1990 and 2000. I then merge this data with my IFLS data on each household's kecamatan in 1997.\footnote{Since the PODES data is comprehensive, there is no attrition when merging the PODES data onto the IFLS data. This approach of merging IFLS data with selected variables from PODES is not novel and has been implemented in the literature \citep{Shrestha2021}.} A visualization of these variables can be found in Figure ~\ref{fig:kinder_numscatter}. 
\begin{center}
	[ Figure ~\ref{fig:kinder_numscatter} here ]
\end{center}
\subsection{Sample Construction and Description}
	\label{sec:sample}
The starting place for my sample is all children \textit{individually} interviewed in the `child' book of the 1997 wave of the IFLS: the `child book' consists of respondents less than 15 years old and 10,356 individuals completed this questionnaire.\footnote{This is Book 5 of IFLS2 -- individuals over 15 years old are interviewed in Books 3A and 3B.} I then add an age restriction, keeping only the children who completed the survey and were between 3 and 9 years old in 1997. With this restriction, I was left with 5,258 individuals. I added an age restriction to ensure that only children who were of kindergarten age in the 1990s (per the years of my instrument) were included in the data.\footnote{To exercise tighter control over age, I broke down individuals in my sample into 4 birth cohorts, those aged 3-4, 5-6, 7-8, 9-10. In terms of the number of households these individuals belonged to, I began with 5,171 unique households and, with the age restriction implemented, I was left with 3,488 unique households.}

Starting with children interviewed in 1997 restricts my sample more than if I had simply looked at children's adult interviews, limiting my variables to their household qualities and individual outcomes.\footnote{For comparison, 20,529 individuals were individually interviewed in the adult survey in the 1997 wave, which was much longer and detailed than the children's individual survey. These individuals were from 7.538 unique households.} Yet, beginning with the childhood individual variables is necessary: I need to get a sense of the pre-kindergarten (or roughly pre-kindergarten) individual characteristics of the children, characteristics such as their health and how often parents took them to the doctor.\footnote{I conduct robustness checks where I relax these specifications to test my hypotheses with fewer covariates and a larger sample size. This narrow sample specification is particularly problematic for my fixed-effects model, a challenge I discuss in greater detail in Section ~\ref{sec:fe} and Appendix ~\ref{app:switching}.} 

Another component of the household survey, in addition to individual interviews, is the household roster--which I use to connect family members to one another and across survey waves. The roster comprehensively documents each member of the household and their basic characteristics--such as their educational attainment and intra-household relationships. I use the rosters to link the 5,258 children from above to 1) their mother and, by extension, 2) their siblings.\footnote{ There is no attrition in this step; children who are individually interviewed are all featured on a household roster, inclusion in the former being a much more stringent restriction than inclusion in the latter.} I then link the household roster to more detailed household survey observations, such as whether a household has electricity, the size of the household, and the household total expenditure. Finally, I merge these households with the IFLS community survey, which contains information of community characteristics such as population and the number of elementary schools in a community. By replacing communities that were not surveyed with appropriate averages, I ensure there is no attrition with this step.\footnote{ Specifically, I am working with 3 community-level variables: elementary schools per 10,000 people, junior high schools per 10,000 people, and senior high schools per 10,000 people (each observed in 1997, 2000, 2007, and 2014). If an individual's community was not interviewed and thus these variables are missing, I replace them with the mean for their kecamatan, conditional on urban/rural status. If there is no data for the kecamatan, I replace their missing observations with the mean for the province, conditional on urban/rural status.}

Thus far, I have individual, household, and community data for approximately when individuals went to kindergarten; however, in order to evaluate medium- and long-term outcomes, I need to match these early childhood observations with later-life observations. First, I incorporate household and community data from 2000 and 2007 in order to absorb exogenous shocks to a child's situation as they grow. For example, I continue to track household expenditure levels and size over time. From 1997 to 2000, among the 5,258 children in my sample, only 80 were lost to attrition (1.5\% attrition rate for this step). From 2000 to 2007, only 130 were lost to attrition (2.5\% attrition rate for this step). These attrition rates are very low because I am relying only on the very broad restriction that individuals appear on a household roster--which I use to merge household data onto individual observations.

Finally, I need the educational outcomes of individuals, which are taken from the 2014 wave of the survey--when all of the individuals are over the age of 18. Therefore, for this restriction, I require that all individuals in my sample were interviewed \textit{individually} in 2014.\footnote{ This time as adults, a designation which occurs when an individual is 15 and above for the purposes of the IFLS.} From 2007 to 2014, 1,357 were lost to attrition (26.8\% attrition rate)--resulting in 3,691 individuals. For context, 36,391 individuals (aged 15+) were interviewed individually in 2014.

Lastly, I require that there are no missing observations for \textit{any} of the covariates used in my non-fixed effects regression specification. This leads to the attrition of 533 individuals, so that my final sample consists of 3,158 individuals (14.4\% attrition rate). Summary statistics of my sample for the variables of interest for my sample can be found in Table ~\ref{table:summary_table}. More information on attrition can be found in the appendix and in Table ~\ref{table:attrition_table}.
\begin{center}
	[ Table ~\ref{table:summary_table} here ] \\
\vspace{-0.1in}
	[ Table ~\ref{table:attrition_table} here ]
\end{center}
Clearly, there is some association between 1) kindergarten and urban/rural status and 2) educational outcomes as well as household wealth or resources. There is also clearly enough variation in my core variables--most of the sample did not attend kindergarten (39\% did), and the sample average for completed years of education is 10.97, approximately 1 year short of the completion of high school), while the median number of years of education completed is 12. 

To give a sense of how representative the sample is of the whole Indonesian population, 95\% of the sample completed elementary school, 81\% completed junior high school, and 62\% completed senior high school. While historical data is difficult to match, pre-pandemic World Bank Data estimates that Indonesian students average 12.4 years of schooling and 37\% of 3-6 years were enrolled in preprimary education in 2018 \citep{Afkar2020}. 2022 OECD data shows that 57.5\% of individuals 25-34 years old graduated senior high school, Thus, my sample appears to be broadly aligned with national summary statistics, although it may be that the sample is over-educated relative to the total Indonesian population.}

\section{Methodology}
	\label{sec:method}
As discussed in the Section ~\ref{sec:lit_rev}, there are significant empirical challenges to isolating the causal effect of early childhood investments on later-life outcomes. One common challenge is selection bias, particularly in Indonesia, where kindergarten is not compulsory and in the 1990s kindergartens were almost entirely private.\footnote{See Figure ~\ref{fig:kinder_numscatter}. Based on the PODES data, in 1990, 95.06\% of kindergartens were private. That share grew to 96.63\% in 2000.} Clearly, kindergarten included a significant opportunity cost and required a significant investment of resources by a family to pay tuition and physically transport their young child to a classroom when it’s not required by law. To meet this challenge, I employ three empirical strategies: ordinary least squares (OLS) estimation, mother fixed-effects (FE), and then instrumental variable (IV) estimation.\footnote{I also use Logit regressions to support my primary analysis, such as to examine selection into attrition and kindergarten; for a description of that method, see Appendix ~\ref{app:logit}.} I believe that selection bias into kindergarten affects the OLS analysis through omitted variable bias--which motivates my use of mother fixed-effects and IV to counter.

\subsection{OLS}
My starting point is a basic OLS estimation of the effects of kindergarten, using the following model:
\begin{gather}
	Y_{if} = \beta_0 + \beta_1\text{KINDER}_{if} + \beta_2\mathbf{X}_{if} + \beta_3\mathbf{Z}_f + \beta_4\mathbf{K}_f + \epsilon_i
\end{gather}
where $Y_{if}$ is the outcome variable for individual $i$ in family $f$, $\text{KINDER}_{if}$ is a dummy variable for kindergarten attendance for individual $i$, $\mathbf{X}_{if}$ is a vector of individual-level variables, $\mathbf{Z}_f$ is a vector of household-level controls, and $\mathbf{K}_f$ is a vector of community-level controls. Our coefficient of interest is $\beta_1$, which represents the effect of kindergarten attendance on the educational outcome.

Critical to my OLS approach is that I control for the two forces driving household human capital decision-making: 1) household expenditure/resources and 2) the importance placed on education/human capital within the household.\footnote{Examples of the former include the natural log of household per-capita expenditure for every available year, tracking household expenditure across the educational career of an individual (this exogenous variable ostensibly absorbs regional economic shocks as well) and whether a household has electricity, also captured over time. Examples of the latter include a mother's or household head's years of completed education, whether an older sibling attended kindergarten, and the number of visits to a doctor by each child. These controls are implemented in conjunction with more basic individual and community controls. For example, in my analyses I control for the number of times a mother has taken an individual child to the doctor in the last few months. This, when weighed against evaluations of the health of the child, may be a proxy for that mother's willingness or desire to invest in the human capital of the child -- after all, ensuring health in early childhood has been demonstrated to greatly enhance human capital accumulation \citep{Attanasio2020}.}

As part of my OLS estimations, I also employ province fixed-effects--as it is clear that there is a heterogenous component of the relationship between kindergarten and education across provinces, as demonstrated in Figure ~\ref{fig:kinder_prov}.\footnote{This province panel variable is more general than the mother identification variable; therefore, while this fixed-effects is less powerful than mother fixed-effects, it also does not restrict my sample--as Figure ~\ref{fig:scatter_switching} makes clear, there is sufficient variation in every province so that my `switching sample' (see below) is the same as my total sample).} Finally, I use cluster-robust standard errors for the OLS estimations; thus, I cluster standard errors at the kecamatan--the equivalent of a county in Indonesia--level, making my estimates heteroskedastic-robust. As I discuss in Section ~\ref{sec:results} and Appendix ~\ref{app:ols_post}, I use a variety of post-estimation analysis to ensure my results are robust and the assumptions of OLS hold up under scrutiny.
\begin{center}
	[ Figure ~\ref{fig:kinder_prov} here ]
\end{center}

\subsection{Mother Fixed-Effects}
	\label{sec:fe}
For as many variables I can control in the above OLS model, there are critical household variables that I simply do not have data for--in particular, related to unobservable characteristics such as the emphasis placed on education within a household. Therefore, there will always be omitted variables correlated with \textit{both} kindergarten attendance and the educational outcome variables in the OLS model. I hypothesize this would result in a upwards bias in the estimate of the coefficient for kindergarten attendance, as powerful positive explanatory variables for educational outcomes correlated with kindergarten attendance are omitted: thus, by absorbing these omitted variables into the fixed-effects term, I expect non-biased estimates using the mother fixed-effects model.

This omitted variable bias motivates an oft-employed approach to examining the effects of early childhood interventions: family fixed-effects \citep{Currie1993,Garces2000}.\footnote{The stronger instance of family fixed-effects is the twin studies I described in the Section ~\ref{sec:lit_rev}.} In particular, I employed mother fixed-effects, an approach that is powerful because it immediately controls for \textit{all} household-level variables. FE accomplishes this feat by comparing children of the same mother. The fixed-effects approach estimates the following equation:
\begin{gather}
	Y_{if} = \beta_0 + \beta_1\text{KINDER}_{if} + \beta_2\mathbf{X}_{if} + \beta_3\mathbf{M}_{ft} + \mu_f + \epsilon_i
\end{gather}
Note that the variables constant within the family from equation (1) are replaced with a single fixed effects term, $\mu_f$, which captures \textit{all} mother characteristics--including all those that were not observed as part of the IFLS. Fixed-effects, however, is not a panacea; it only controls for all \textit{time-invariant} characteristics of the mother (rooted in 1997--the initial point of my sample). Therefore, I need to incorporate a vector of time-varying mother/household characteristics, represented by the term $\mathbf{M}_{ft}$.\footnote{These characteristics include economic or health shocks to the mother/household, the household economic status, the household structure (i.e., whether it remains a two-parent household or its size), and community-level shocks such as natural disasters.} It is possible that in reality a mother could treat their children differently—accordingly I’ve attempted to control for some pre-kindergarten characteristics that may signal favoritism, focusing on such indicators identified by previous work \citep{Garces2000}. 

To avoid the bias of unobserved mother favoritism, I include a variable representing the number of times a parent took a child to the hospital as well as the general health of the child in 1997. This is based on a hypothesis that parents concentrate investment into the child they believe has the best chance of success--so the more often a parent takes a child to the doctor, health held constant, the likelier they may also be to send that child to kindergarten \citep{Duflo2011}. 

It is also possible that there are spillover effects from one child attending kindergarten on successive children--something I attempt to control for by including a dummy variable for whether an older sibling had already attended kindergarten. Lastly, it is possible that a family has a different quantity of resources at its disposal when one child is of kindergarten age than when another child is--something I control for by employing controls for time-varying household characteristics captured by household expenditures. The latter is a particular concern, as fixed-effects controls for all unobserved \textit{time-invariant} characteristics. I alleviate this concern by controlling for age.

Fixed-effects introduces another complication; in order to run a regression for a specific family $f$, there must be variation in the explanatory variable, introducing a two-step requirement to be included in the fixed-effects model specification. These requirements are that 1) each mother has more than one child in our sample and 2) there is variation between a mother's children in kindergarten attendance. This significantly restricts our sample--as seen in the diagnostic graph Figure ~\ref{fig:scatter_switching}.\footnote{17.48\% of the sample are from households with exactly one child. Only 7\% of my total sample qualifies as part of a switching household according to these requirements--meaning there is a sample loss of 93\% just for the fixed-effects sample specification.}
\begin{center}
	[ Figure ~\ref{fig:scatter_switching} here ]
\end{center}
These requirements for my fixed-effects specifications thus distinguish between two types of families already included in my sample: 1) ``switching" families and 2) ``non-switching families". \citep{Miller2023} ``Switching" families are included in the fixed-effects model and ``non-switching" families are not. Often scholars are concerned about selection bias into the ``switching" designation. I find the ``switching" sample has some puzzling characteristics when compared to the total sample. There is no significant positive selection bias, as the only significant predictor of whether a family is a ``switcher" is household size, which I do not find to be significant in explaining educational outcomes. However, selection into kindergarten attendance appears to be a drastically different process--something I discuss in greater detail in Appendix ~\ref{app:kinder_sel} and Appendix ~\ref{app:switching}. 

In brief, while household characteristics function similarly in determining kindergarten attendance for switching and non-switching families, individual pre-treatment characteristics--the only category of covariates I can control for in fixed-effects--appear to function in the opposite direction. For example, for non-switching families there is a significant negative effect of whether an older sibling attended kindergarten on selection into kindergarten. Additionally, the switching sample is so small as to render any OLS or IV estimates of the effect of kindergarten restricted to the switching sub-sample insignificant. 

\subsection{Instrumental Variable (IV) Estimation}
Each of my first two empirical approaches--OLS and fixed-effects--have shortcomings. First, OLS merely suggests an association between kindergarten attendance and educational outcomes. Critically, it suffers from omitted variable bias because of a host of unobservable variables. Second, as powerful as fixed-effects is for controlling household factors, it introduces significant sample restriction and selection bias into the switching sample.

This motivates me to use instrumental variable (IV) estimation methods to remedy the bias introduced by the endogenous regressor, kindergarten attendance. I develop two instruments, both taken from the Village Potential Statistics (PODES) waves of 1990 and 2000; PODES contains explicit data about the presence of kindergartens in villages--see Section ~\ref{sec:podes} for more information. I aggregate population and kindergarten across each kecamatan and then merge based on the kecamatan an individual or household is registered under in the 1997 wave of the IFLS, creating a population-weighted average of kindergartens for each kecamatan. My first instrument consists of total (private + public) kindergartens per 10,000 people in each kecamatan in 1990 and my second instrument is the same measure in 2000.\footnote{In Appendix ~\ref{app:iv_robust} I explore alternative instruments, such as comparing just private to just public kindergartens, taking an average of the 1990 and 2000 measures, and percent change in the presence of kindergartens.} While I use the total number of kindergartens, the data clearly reflects it is private kindergartens driving expansion in the total number of programs between 1990 and 2000, as seen in Figure ~\ref{fig:kinder_numscatter}.
\begin{center}
	[ Figure ~\ref{fig:kinder_numscatter} here ]
\end{center}
There are different estimators for IV estimation; I employ the Generalized Method of Moments (GMM) estimator. In cases of over-identification, GMM does not reduce multiple instruments into one matrix and therefore is a more efficient estimator. It also gives me greater flexibility in conducting over-identification tests. The employment of an instrument requires a robust defense--both intuitively and empirically. An instrument needs to be predictive of kindergarten attendance while having no impact on educational attainment when relevant factors are controlled for \citep{Levitt2002}.

	\subsubsection{Instrument Strength} 
First, intuitively the two instruments will each be very strong: we would expect that the number of kindergartens in a locality, all else being equal, would have an effect on whether a random student from that locality attends kindergarten. This theoretical expectation is backed up empirically by the results of the first-stage regression, found in Table ~\ref{table:iv_first_stage}. One of the advantages of GMM is that it employs the two instruments separately; therefore, to demonstrate instrument strength, I regress each instrument on kindergarten attendance separately. While it does appear that the number of kindergartens per 10,000 people in 1990 is a stronger instrument, clearly the corresponding 2000 variable is significantly strong.\footnote{I use a basic F-Test to reject the null hypothesis that either coefficient is equal to 0 in the fully-specified model, each with a p-value of 0.0000. More broadly, the `rule of thumb' for first-stage regressions is to look for an F-Statistic greater than 10; for the 1990 instrument, the F-Statistic is 136.48 in the fully-specified model, and for the 2000 instrument, it is 121.78.} Since the regressor being complemented by the instrument and the instrument are the same across models--only the outcome variable changing--I only need to check instrument strength once.
\begin{center}
	[ Table ~\ref{table:iv_first_stage} here ]
\end{center}
	\subsubsection{Instrument Validity}
Second, instrument validity requires a more finely-tuned theoretical proof, as well as more detailed empirical evidence. I suspect that these instruments do not exert an effect on educational outcomes \textit{when} other community factors are controlled for. In particular, I control for the number of elementary schools per 10,000 people in each locality in 2000, junior high schools in 2007, and senior high schools in 2014; controlling for the presence of these schools, I theorize, isolates the effect of the supply of kindergartens to only kindergarten attendance. Thus, any effect that the supply of kindergartens has would be mediated through kindergarten attendance. This broad theoretical argument would apply for any of the outcome variables I research in this paper.\footnote{Not all kindergartens are the same, and I naturally expect some kindergarten programs to be of a higher quality than others. Neither PODES nor IFLS has data on the \textit{details} of kindergarten programs and their quality, so this would be an area for further research, and one I discuss in Section ~\ref{sec:conclusion}.} 

For the empirical evidence of validity, I rely on Hansen's J Test, or the over-identifying restrictions test. Two things allow me to run this test: 1) I have two instruments for one endogenous regressor and 2) I employ the GMM estimation method. I then run the over-identifying restrictions test for every model, as the outcome variables alter the calculation of the test's F-Statistic. Overall, the results of this test are promising: I fail to reject the null hypothesis that at least one of the two instruments is not valid for all models except elementary school completion, junior high school completion, school attendance for grades 3-9, and stay-on decision making after grade 6.\footnote{Elementary school completion, and school attendance for grades 3-6 do not concern me very much, as there was not enough variance in the outcome to begin with and model results are thus not very significant.} This largely corroborates the theoretical argument I outlined above.



}

\section{Results}
	\label{sec:results}
\def \data #1{/Users/danielposthumus/thesis_independent_study/work/writing/rough_draft/analysis/#1}

I organize my results by outcome variable: each sub-section will include my OLS, fixed-effects, and instrumental variable estimates for the effects of kindergarten on that outcome variable. I have included results and discussion of an analysis of selection into kindergarten attendance in Appendix ~\ref{app:kinder_sel}. 

\subsection{Years of Education}

First, I examine years of education as the outcome variable. All of the individuals in my sample were between 20 and 27 years old by 2014, when the data on years of education completed was collected, ensuring that every individual had the opportunity to finish high school and advance into college before the variable was collected.\footnote{When focusing on outcomes such as school attendance and stay-on rates, I exclude years of schooling beyond the 14th because, although every individual would have had the opportunity to complete high school and enter into college, the data on college years is incomplete.} I employ four model specifications for the OLS and fixed-effects, as can be found in Table ~\ref{table:full_results}.\footnote{I do not include the point estimates of the coefficient for every covariate for conciseness. Also note that I incorporate province fixed-effects in Model (3), alongside the other community controls that are only included in Model (3).}
\begin{center}
	[ Table ~\ref{table:full_results} here ]
\end{center}
To begin, there is variation in the years of education completed; in my sample, 37.3\% of individuals completed exactly, 38.19\% of individuals completed fewer, and 34.51\% of individuals completed more than 12 years of education. The distribution of years of education completed is nearly identical within the switching sample.\footnote{Within the switching sample, 37.56\% completed exactly 12 years of education, 36.65\% completed fewer than 12 years of education, and 35.79\% completed more than 12 years of education.} Overall, I find significant, positive effects of attending kindergarten. For the OLS specifications, the inclusion of more controls reduces the magnitude of the significant positive effect of kindergarten until kindergarten’s effect is estimated to be adding 0.71 years of education, in the fully specified model. Other significant positive predictors of years of education include household wealth and the mother's level of education, while being male has a significant negative effect on the years of education completed. The mother fixed-effects model, however, has unclear and insignificant results: with fixed-effects, I find none of the included covariates exhibit any significant effect on the years of education completed. I suspect that this is because of the bias introduced by the high rates of attrition into the switching sample used to fit the fixed-effects model--something I discuss in greater detail in Appendix ~\ref{app:switching} and which a robustness check, discussed in Appendix ~\ref{app:switching_ols}, confirms. 

Next, I employ Instrumental Variable (IV) estimation, the results of which can be found in Table ~\ref{table:iv_main_results}. The IV estimates for the effects of kindergarten is greater in magnitude than that for the OLS and fixed-effects models for all specifications. In particular, the fully specified model estimates that kindergarten adds 1.89 years of education--compared to 0.71 years in the fully specified OLS model. Two coefficients are very different in these results: the dummy variables indicating whether a household was urban in 1997 and the dummy variable for whether the mother as well as the father are both present in a child’s household in 1997. Whereas each of these coefficients were statistically insignificant in the OLS and fixed-effects models, they are positive and significant in the IV estimation.\footnote{All covariates in this model are the same as the ones I described above; thus, the community controls in this model are the elementary schools per 10,000 people in 2000, junior highs per 10,000 people in 2007, and senior highs per 10,000 people in 2014.} 
\begin{center}
	[ Table ~\ref{table:iv_first_stage} here ]
	
	[ Table ~\ref{table:iv_main_results} here ]
\end{center}
This increase in the coefficient magnitude from OLS to IV is surprising, and runs counter to my initial hypothesis that omitted variable bias in the OLS model introduced an upward bias because of omitted explanatory variables I would expect to be correlated with both kindergarten attendance and greater educational outcomes. One hypothesis for this increase may be that \textit{accessibility} is a more significant explanatory variable of kindergarten attendance, and by extension schooling, than I would expect. IV exploits the instrument to focus on kindergarten's effect on the portion of the population whose human capital investment decisions were affected by the instrument--the number of kindergartens in an area. Given the number of private kindergartens then, we may imagine a large portion of the population who attended kindergarten would have attended kindergarten even if they had to go to another sub-district; they may have the resources to do so. We can imagine this same population is likelier to complete high school regardless of kindergarten attendance. On the other hand, the portion of the population who would only be attending kindergarten if there were a large number in \textit{their} sub-district would be less wealthy and thus, perhaps, kindergarten would have a greater effect for this subsample. IV, by design, may be more swayed by this latter group, and thus result in higher estimates of the effect of kindergarten. This gap between estimates merits further closer investigation. This hypothesis would apply to other schooling variables as well.

\subsection{School Completion}

Completed years of education is a long-term outcome, and I can break it up into the completion of the three levels of schooling: elementary, junior high, and senior high school.\footnote{In Indonesia, elementary school ends after the completion of the 6th grade, junior high after the completion of the 9th grade, and senior high after the completion of the 12th grade.} In this section I analyze the effects of kindergartens where dummy variables indicating whether a child completed each of these levels of schooling are the outcome variables—allowing me to analyze heterogeneity in kindergarten’s effects over the life cycle and examine whether there is fade-out. Results of this analysis can be found in Table ~\ref{table:full_complete}.
\begin{center}
	[ Table ~\ref{table:full_complete} here ]
\end{center}
School completion is a slightly challenging outcome to analyze. First, for elementary school completion there is almost no variance--only 5.29\% of the sample failed to complete elementary school. I suspect in large part because of this lack of variance in elementary school completion in the sample, the OLS model finds no significant effects of kindergarten on the completion of elementary school. On the other hand, 18.68\% failed to complete junior high school and 38.19\% failed to complete senior high school.

There is significant heterogeneity across the three model types: OLS, mother fixed-effects, and IV estimation. Focusing on junior and senior high school, the OLS model finds significant positive effects for junior and senior high completion. Similar to the estimates for years of education, fixed-effects once again finds no significant effect of kindergarten attendance. The IV estimates, on the other hand, suggest fadeout; there is a very strong significantly positive effect of kindergarten attendance on the completion of junior high school that `fades' to insignificance for senior high school. We can focus on even more granular outcomes than school completion by focusing on specific grades and their rates of completion.

Importantly, the estimates for the covariates' coefficients are very similar across the OLS and IV estimates. Household expenditures become increasingly important over the life cycle (particularly the most recent observation in 2007), and the mother's years of education becomes more significant over time as well. This stands in contrast to the declining effects of kindergarten post-junior high found above, and that will also be found below in Figure ~\ref{fig:inschl_reg}, which focuses on school attendance--the next outcome I will analyze.\footnote{This difference between junior and senior high suggests that the larger effect of kindergarten on junior high is not merely a function of greater variance in the outcome variable, since we do not see a similarly increased estimate for senior high completion.}

\subsection{School Attendance and Stay-On Decision}
The next, more granular outcome to examine is grade-by-grade school attendance--captured by a dummy indicating whether a child completed every single grade.\footnote{ I focus on grades 1-14, for reasons described above.} In Figure ~\ref{fig:inschl_reg}, I have plotted the attendance rates themselves by grade and then the effects of attending kindergarten by grade. The vertical lines on the attendance rate graph indicate the 6th, 9th, and 12th grades, which are the final grades for elementary, junior high, and senior high school, respectively--note the steep drop-offs in attendance following the conclusion of each grade.
\begin{center}
	[ Figure ~\ref{fig:inschl_reg} here ]
\end{center}
We can make a few observations: first, as I mentioned above, there is nearly no variation in school attendance throughout primary school--particularly for children who attend kindergarten. However, the estimated marginal significance of kindergarten's effects throughout grades 1-6 (consisting of elementary school) suggests the gap between kindergarten and non-kindergarten children is explained by other factors. The estimated effect of kindergarten is greatest in junior high school--for the OLS and IV estimates. After junior high, however, the effects of kindergarten fade to marginal significance once again and become nearly exactly zero after high school. We once again have preliminary evidence of fadeout.

Examining school attendance over time has a critical advantage--the sample is kept constant across time. The disadvantage of this approach, however, is that once a child stops attending school, the only possible values for the remaining attendance dummy variables is 0. Thus, I corroborate analyzing school attendance with looking at `stay-on'; a dummy variable conditional on a student already being in a grade, taking on a value of 1 if they \textit{continue} in school, and 0 if they exit school after that grade.\footnote{For the stay-on analysis, the sample varies across grades since if a student had dropped out after the 4th grade, say, their assigned value for the 6th grade stay-on variable is missing and not zero. For school attendance by grade, on the other hand, the attendance dummy for 6th grade in this situation would be zero, not missing.} The results of that analysis can be found in Table ~\ref{table:full_stayon}.
\begin{center}
	[ Table ~\ref{table:full_stayon} here ]
\end{center}
These results provide strong evidence of fade-out. The OLS estimates suggest that kindergarten has a strongly significant positive effect on whether a student stays on after the 6th grade, and a weaker effect corresponding to the 9th and 12th grades. The fixed-effects model suggests there is no effect, though there is some very weak suggestion of a negative effect for whether a student stays on after the 12th grade. 

Finally, the IV estimates--my primary empirical method--show a very strong significant positive effect for 6th grade and absolutely insignificant effects for staying on after 9th and 12th grades. This corroborates the evidence from the school attendance analysis: kindergarten may not have an effect on completing elementary school, but it may lead children to attend junior high school, after which kindergarten has no \textit{enduring} effect. Interestingly, there is no evidence that the number of schools in a locality has any effect on whether a student stays on--suggesting that supply side forces are not powerful in explaining educational attainment, though this is outside the scope of this research and motivates further work.

\subsection{Cognitive Test Scores}

Thus far, my educational outcomes have been different measures of educational attainment; however, as I make clear in Section ~\ref{sec:intro}, in Indonesia, student \textit{learning} is not keeping up with rapid advances in school attendance. I am not able to match standardized test scores to individuals in the dataset; however, the IFLS has a book in which interviewees are subject to a basic cognitive test. This cognitive test has two versions: one for children aged 7-14 and the other for respondents aged 15-24. I standardize each individual's test score according to their age group's performance on the test to ensure I am only comparing like-aged individuals. Each test consists of two parts: the first is basic visual recognition of shapes and patterns. An example page of this test from the 15-24 version in 2000 can be seen in Figure ~\ref{fig:ek_shapes}.
\begin{center}
	[ Figure ~\ref{fig:ek_shapes} here ]
\end{center}
The second portion consists of basic arithmetic problems. For the 7-14 aged children, these are just math problems while for the 15-24 age group they are more involved word problems. An example page of this test from the 7-14 test in 2000 can be seen in Figure ~\ref{fig:ek_math}.\footnote{These tests are not perfect and there may be measurement error; however, all covariates except for kindergarten attendance (in particular, the mother's years of education and household expenditure over time) behave very similar for the models with schooling outcomes as for this model with cognitive test performance outcomes, so there is not something systematically biasing results across covariates/explanatory variables.}
\begin{center}
	[ Figure ~\ref{fig:ek_math} here ]
\end{center}
I then use individuals' performance on these tests in 2000, 2007, and 2014 to effectively track their cognitive abilities over time.\footnote{I create a sub-sample for this analysis, restricting the sub-sample to only those individuals who completed all 3 tests in order to effectively compare tests over-time.} The results of that analysis can be found in Table ~\ref{table:full_ek}.\footnote{All individuals completed the same cognitive test; therefore, I have added age fixed-effects to account for the differing intercept based on age that we would expect.}
\begin{center}
	[ Table ~\ref{table:full_ek} here ]
\end{center}
These results are mixed; the OLS estimates of the coefficient of kindergarten attendance tell a clear story of fade-out. Kindergarten has a strong positive effect on the 2000 scores, then a still significant and positive, although weaker, effect on 2007 scores, before fading to insignificance for the 2014 scores. The fixed-effects results suggest insignificant effects, and the IV--most perplexingly--also suggest insignificant effects. 

The IV estimates tell two very different stories of kindergarten’s effect: one for cognitive performance, where kindergarten has no significant effect, and one for educational attainment, where kindergarten has a very significant positive effect. This divergence would confirm fears outlined in Section ~\ref{sec:intro} that increasing enrollment and school completion may not be translating to better learning and effective human capital formation for students. 

We can also look at change in cognitive performance across various years: perhaps kindergarten resulted in increased performance from 2000 to 2007, even if performance remained low. First, we can look at percent change in test performance from 2000 to 2007. For both un-standardized and standardized percent change (standardized according to age), kindergarten has a positive insignificant effect. Second, looking at change from 2007 to 2014, kindergarten’s effect’s decreases while remaining insignificantly positive, for both standardized and un-standardized change. It is tempting to suggest these results suggest fadeout; however, there was no significant effect of kindergarten to begin with, preventing me from interpreting these results with any significance or weight. 

The lack of significant results using IV estimation is made more credible because the IV estimates were so large and statistically significant for the educational attainment variables. If there was some upward bias of the IV method in estimating kindergarten effects, then they would also be at work here--but we see no significant estimate of the effect of kindergarten. This divergence is also critical because the covariates previously demonstrated to be significant and critical to explaining educational attainment--namely household expenditure in 2007 and the mother's years of education--are still very significantly positive in explaining cognitive test performance. I will discuss the interpretation and significance of these results in greater detail in Section ~\ref{sec:conclusion}. 


 }

\section{Conclusion}
	\label{sec:conclusion}
\input{\writing{conclusion.tex}}
\clearpage

\clearpage
\bibliographystyle{ecta}
\bibliography{bibliography.bib}

\clearpage
\newgeometry{left=0.25in,right=0.25in,top=0.25in,bottom=0.25in}

\section*{Figures and Tables}
\addcontentsline{toc}{section}{Figures and Tables}
\input{\writing{figures.tex}}
\clearpage

\restoregeometry

\begin{appendices}
	\input{\writing{appendix.tex}}
\end{appendices}


\end{document}
