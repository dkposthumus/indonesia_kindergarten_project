My research question focuses on one determinant of educational outcomes, kindergarten attendance, placing my research in the literature of human capital investments and early childhood interventions. I will begin my review with a summary of the human capital literature's findings on the determinants of educational outcomes. I will describe two categories of determinants, 1) genetic endowments and 2) human capital investments. I will then lay the groundwork of my theoretical approach, based on the work of Cunha and Heckman (2007), in which childhood is multi-staged and skills learned in one stage dynamically interact with skills learned in another stage. I will then conclude with a review of previous empirical findings regarding the effects of kindergarten/preschool, with a particular focus on how these findings--concentrated in developed countries--may or may not translate to developing countries such as Indonesia.

\subsection{Determinants of Educational Outcomes}
Broadly, we can categorize determinants of educational outcomes into two: 1) genetic endowments--often referred to as `ability'--and 2) human capital investments. In the first category, we can think of genetic endowment as something fixed at birth; it is difficult to measure, but the idea is that something like an individual's intelligence quotient (IQ) is indicative of `natural talent', \textit{independent} of their lived experiences. There is strong evidence that endowments of talent, ability, or intelligence are positively associated with schooling--one influential estimate of the correlation of intelligence and schooling is 0.5 \citep{Johnson2005}. One study of twins in Australia found that between 50 and 65\% of variance in schoolings between sets of twins could be explained by genetic endowments, a finding in line with preceding estimates \citep{Miller2001}. Behrman and Taubman (1989) estimate the share is 90 percent. One longitudinal study of about 70,000 children in the United Kingdom found the correlation between a child's latent, or natural, intelligence and their standardized test scores 5 years later to be 0.81 \citep{Deary2007}.

Studies examining the role of natural endowments in human capital accumulation, however, harbor fundamental flaws. First, they rely on a convoluted causal story; there is evidence that more time in school increases latent intelligence, creating a virtuous cycle. Additionally, the relationship between genetics and schooling is very heterogenous--there is evidence, for example, that as limits on women's opportunity to attain education decreased in the 1970s and 1980s, the relationship between genetics and schooling strengthened as constraints on access to education eased \citep{Herd2019}. The dominance granted genetics by the estimates described above may be wildly divergent in contexts where there are either greater constraints on access to education, or (as described below) the opportunity cost of education is so high as to entirely distort human capital investment decision-making.

A second issue with studying the effects of genetic endowments is that `latent talent' is very difficult to measure--hence the predominance of twin studies, since twins’ genetic variation is significantly easier to identify \citep{Miller2001}. For measuring genetic endowment outside of the twin context, however, some scholars have pointed out that standardized tests of intelligence resemble classroom work, putting into question the ability of these tests to distinguish between the endowments independent of the upbringing of a child and the results of that upbringing \citep{Johnson2009}. Because of these challenges, I do not make studying genetic endowment a central part of my research.\footnote{Mother fixed-effects will ostensibly control for some genetic endowment, albeit imperfectly since my study is not one of twins. While in theory I could go further and incorporate variables collected during a pregnancy (attempting to control for genetic endowments traced back to the ``fetal origins hypothesis"), such data would prohibitively restrict my sample size \citep{Almond2011}. I do attempt to control for \textit{some} genetic variation by including controls for health as well as visits to healthcare (perhaps a child is naturally sickly), although this is only a preliminary step towards understanding the role of genetics in human capital accumulation in Indonesia.}

The second category of determinants of schooling is human capital investment—my research is situated in this category of work. While the genetic endowment of a child is fixed at birth, human capital stock, on the other hand, is anything but fixed, and changes according to investments. We can further divide these investments into two types: 1) human capital investments \textit{within} the household, and 2) human capital investments from \textit{outside} of the household. In this literature review, I focus on investments in early childhood, as my research focuses on one such early childhood intervention--kindergarten. Henceforth, I will refer to human capital investments made \textit{without} the household as early childhood interventions, such as attending Head Start in the United States or kindergarten in Indonesia.

My motivation to focus on early childhood is clear: there is overwhelming scientific evidence that at an early age children's brains are more malleable and receptive to learning skills than at later ages \citep{Cantor2019,Duncan2023}. In particular, evidence has shown that children's brains are most malleable to new learning for the first five or six years of life--fitting into the early childhood education ``window" lasting from birth to the age of 12 \citep{Slegers1997}. Brain development does not occur just in a classroom--neglect as a child results in delays in brain development: during this early childhood window, a child is undergoing dynamic transformations, with human capital investments both within the household and without the household interacting in powerful ways \citep{Perry1997}. We cannot understand the latter (i.e., the effect of kindergarten), without understanding the former.

The motivation for targeted early childhood programs such as Head Start rely on this interaction; some disadvantaged children are born into situations where neglect occurs--resulting in inequity in brain development and human capital between the advantaged and disadvantaged before a child even enters a classroom. Therefore, human capital investments received in the classroom can compensate for a lack of investments made at household for disadvantaged children--promoting equity and efficiency \citep{Heckman2011}. 

\subsection{Theoretical Approaches to Studying Early Childhood Education}
In addition to motivating early childhood interventions, economists have used scientific findings about brain development to create models in which skills learned in early childhood complement skills learned later in childhood or even in adulthood. This dynamic is most importantly explained by the concepts of self productivity and dynamic complementarity posited by Cunha and Heckman (2007). While the dominant assumption of theoretical work prior to 2007 followed Becker (1986) in treating childhood as a single period (with adulthood as a separate, succeeding, period), Cunha and Heckman instead treated childhood as two periods--with skills learned in the one period having a dynamic complementarity with skills in the second \citep{Becker1986,Cunha2007}.

This dynamic complementarity also applies to two different categories of skills, cognitive and non-cognitive--the former of which I focus on. Cunha and Heckman (2007) outline how cognitive and non-cognitive skills complement each other and focusing strictly on cognitive performance--which studies of Head Start largely do--tells an incomplete story. This is a compelling argument, and motivates further research into my question. If kindergarten adds more years of schooling but does not simultaneously improve learning or cognitive outcomes, then those additional years of schooling could nonetheless be generating more positive social outcomes. This may not help to address concerns about human capital formation and Indonesian students' performance on standardized test scores, but would bode well for kindergarten's societal effects.

\subsection{Empirical Findings of the Effects of Kindergarten}
Kindergarten in Indonesia is essentially the equivalent of preschool in the United States; it is not compulsory, with both state-owned and private options (the latter of which is largely responsible for increases in kindergarten attendance, as seen in Figure ~\ref{fig:kinder_numscatter}). Empirical studies of the effects of preschool on educational outcomes fall into two categories: 1) randomized control trials (RCTs) and 2) longitudinal studies \citep{Duncan2013}. My project is the latter, as I track individuals from before they attend kindergarten to until after high school graduation.

I begin my review of the empirical findings on the effects of preschool by examining the US context. Head Start in the United States is the most common subject of preschool studies: it is a program run by the US Federal Government since the 1960s, when it was established as part of the Great Society. Head Start targets low-income children, so that federal guidelines require 90\% of children served by the program come from families below the poverty line \citep{Currie1993}. The other two most popular US programs that have been studied are the Perry Preschool Program and the Abecedarian Project, both of which are intensive and have been implemented at a much smaller scale than the massive Head Start program \citep{Heckman2010,Campbell2002}.  Because the latter two programs are intensive and revolve around very small class sizes, they are favored by scholars employing RCTs \citep{Muennig2011}.\footnote{As an example of the intensiveness of the Abecederian Project, instructors create personalized games and curricula for individual children, and children receive instruction for 5 years. Head Start matches neither the personalized nature nor duration of the Project.} On the other hand, most studies of Head Start rely on longitudinal designs.

Studies of Head Start have shown mixed results regarding its effect on later-life outcomes. In 1993, Thomas and Currie (1993)--using mother fixed-effects, just as I do--found that attending Head Start compared to either 1) attending other preschool programs or 2) not attending preschool at all, had some significant effects on test scores for white and Hispanic children, while program participation had no effect for Black children. For that paper, however, they focused only on \textit{short-term} effects: 10 years later, Thomas, Garces, and Currie (2002) revisited Head Start, this time looking at its effects on \textit{long-term} outcomes. Using mother fixed-effects once again, they found evidence that white children who attended Head Start were likelier to complete high school than their siblings who did not--they also found \textit{some} evidence of the same effect for Black children. When looking at the long-term effects of Head Start, Thomas, Garces, and Currie, in particular, found strong evidence for a critical concept in the early childhood literature, ``fadeout". 

Fadeout is an important concept in the early childhood interventions program; its basic premise is that the positive effects of participating in an early childhood intervention decrease or even approach insignificance as a child ages \citep{Abenavoli2019}. Fadeout is rooted in Cunha and Heckman's multi-stage approach to childhood; in particular, there is evidence that fadeout occurs only when high-quality interventions are not followed by subsequent high-quality educational experiences \citep{Bailey2017,Lee1995,Jenkins2018}. 

Fadeout also reveals significant heterogeneity among preschool programs, as high-quality ones are more durable than lower-quality programs. Clearly, fadeout is far from universal; there is even evidence that preschool leads to reduced adulthood delinquency and crime \citep{Barnett2008}. It is easy to see how this concern directly applies to Indonesia: classroom quality is a primary concern for policymakers as Indonesia has not improved its PISA test scores in recent years despite near-universal completion of primary school \citep{Afkar2020}. These stagnating test scores may be due to a lack of investments in improving teacher quality, the adoption of learning technologies, and increasing the number of teachers to improve teacher-student ratios. Thus, if later-life investments in education are not being made, then improvements in primary school completion aren't being capitalized, in view of Cunha and Heckman's definition of childhood. Therefore, fadeout--which I analyze by comparing the effects of kindergarten on short- vs. medium- and long-term outcomes--is of prime concern for my research.

Work on preschool in the developing context is limited; this makes sense, as developing countries are less likely to have the comprehensive multi-wave longitudinal household surveys like the Panel Study of Income Dynamics (PSID) that make researching programs like Head Start possible, particularly for the favored research design of family fixed-effects, which requires a large number of households, and detailed within-household data. In this regard, the Indonesian Family Life Survey (for more details, see Section ~\ref{sec:data}) provides a unique opportunity.

There are reasons why these effects of preschool interventions might not translate to developing countries \citep{Dean2020}. First, instruction quality might be weaker. Second, the effects of preschool are naturally weighed against the counterfactual--and the counterfactual to early childhood intervention varies widely across contexts.\footnote{In the US, for example, studies have shown declining effects for Head Start--this, however, is largely attributable to the improving counterfactual to participation, rather than changes in the characteristics of the program. We can also see this same pattern occurring in Indonesia, as the gap between children who went to kindergarten and children who did not has been narrowing over time in Figure ~\ref{fig:kinder_binscatter}.} And third, recalling Cunha and Heckman's (2007) conception of childhood, subsequent educational experiences can vary--and thus, through dynamic complementarity of skills across stages of childhood, alter the effects of kindergarten. 

There is strong motivation to study kindergarten in developing countries--one estimate is that 200 million children below the age of 5 in developing countries do not reach developmental potential due to a lack of resources \citep{GranthamMcGregor2007}. There is a great deal of evidence that drastic increases in kindergarten attendance over time are driven by those with greater socioeconomic resources--a phenomenon that the Indonesian data supports, with the predominance of private kindergarten. Overall, there is strong evidence that preschool has very strong positive effects—14 studies of developing countries found, on average, significant positive effects \citep{Behrman2013}. These same studies find significant effects of preschool on schooling and educational achievement. There is also significant evidence of fadeout among these studies.

For more specific estimates of the effects of kindergarten, Dean and Jayachandran (2020) find that, in India, children participating in kindergarten perform 0.8 standard deviations better on cognitive tests than their peers, and that this advantage is persistent even if it decreases over time. They find no impact on socio-emotional development, however. The type of kindergarten a child attends also matters; one study in Ghana found that children who attended \textit{private} kindergarten performed better than their peers who did not attend kindergarten as well as their peers who went to \textit{public} kindergarten \citep{Pesando2020}.

One notable study of preschool in Uruguay, by Berlinski et al. (2008), closely mirrors my research design. The authors found significant positive effects of preschool participation using family fixed-effects; by the age of 15, for example, children who participated in kindergarten completed an extra 0.8 years of education. They focused on grade repetition--i.e., students failing a grade and having to stay there to try again--as the transmission channel for the effects of preschool. Grade repetition, on the other hand, is rare in Indonesia--and it does not help to resolve the Indonesian tension between outcomes in schooling and outcomes in learning and skills \citep{Berlinski2008}.