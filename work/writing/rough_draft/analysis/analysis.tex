\def \data #1{/Users/danielposthumus/thesis_independent_study/work/writing/rough_draft/analysis/#1}

I organize my results by outcome variable: each sub-section will include my OLS, fixed-effects, and instrumental variable estimates for the effects of kindergarten on that outcome variable. I have included results and discussion of an analysis of selection into kindergarten attendance in Appendix ~\ref{app:kinder_sel}. 

\subsection{Years of Education}

First, I examine years of education as the outcome variable. All of the individuals in my sample were between 20 and 27 years old by 2014, when the data on years of education completed was collected, ensuring that every individual had the opportunity to finish high school and advance into college before the variable was collected.\footnote{When focusing on outcomes such as school attendance and stay-on rates, I exclude years of schooling beyond the 14th because, although every individual would have had the opportunity to complete high school and enter into college, the data on college years is incomplete.} I employ four model specifications for the OLS and fixed-effects, as can be found in Table ~\ref{table:full_results}.\footnote{I do not include the point estimates of the coefficient for every covariate for conciseness. Also note that I incorporate province fixed-effects in Model (3), alongside the other community controls that are only included in Model (3).}
\begin{center}
	[ Table ~\ref{table:full_results} here ]
\end{center}
To begin, there is variation in the years of education completed; in my sample, 37.3\% of individuals completed exactly, 38.19\% of individuals completed fewer, and 34.51\% of individuals completed more than 12 years of education. The distribution of years of education completed is nearly identical within the switching sample.\footnote{Within the switching sample, 37.56\% completed exactly 12 years of education, 36.65\% completed fewer than 12 years of education, and 35.79\% completed more than 12 years of education.} Overall, I find significant, positive effects of attending kindergarten. For the OLS specifications, the inclusion of more controls reduces the magnitude of the significant positive effect of kindergarten until kindergarten’s effect is estimated to be adding 0.71 years of education, in the fully specified model. Other significant positive predictors of years of education include household wealth and the mother's level of education, while being male has a significant negative effect on the years of education completed. The mother fixed-effects model, however, has unclear and insignificant results: with fixed-effects, I find none of the included covariates exhibit any significant effect on the years of education completed. I suspect that this is because of the bias introduced by the high rates of attrition into the switching sample used to fit the fixed-effects model--something I discuss in greater detail in Appendix ~\ref{app:switching} and which a robustness check, discussed in Appendix ~\ref{app:switching_ols}, confirms. 

Next, I employ Instrumental Variable (IV) estimation, the results of which can be found in Table ~\ref{table:iv_main_results}. The IV estimates for the effects of kindergarten is greater in magnitude than that for the OLS and fixed-effects models for all specifications. In particular, the fully specified model estimates that kindergarten adds 1.89 years of education--compared to 0.71 years in the fully specified OLS model. Two coefficients are very different in these results: the dummy variables indicating whether a household was urban in 1997 and the dummy variable for whether the mother as well as the father are both present in a child’s household in 1997. Whereas each of these coefficients were statistically insignificant in the OLS and fixed-effects models, they are positive and significant in the IV estimation.\footnote{All covariates in this model are the same as the ones I described above; thus, the community controls in this model are the elementary schools per 10,000 people in 2000, junior highs per 10,000 people in 2007, and senior highs per 10,000 people in 2014.} 
\begin{center}
	[ Table ~\ref{table:iv_first_stage} here ]
	
	[ Table ~\ref{table:iv_main_results} here ]
\end{center}
This increase in the coefficient magnitude from OLS to IV is surprising, and runs counter to my initial hypothesis that omitted variable bias in the OLS model introduced an upward bias because of omitted explanatory variables I would expect to be correlated with both kindergarten attendance and greater educational outcomes. One hypothesis for this increase may be that \textit{accessibility} is a more significant explanatory variable of kindergarten attendance, and by extension schooling, than I would expect. IV exploits the instrument to focus on kindergarten's effect on the portion of the population whose human capital investment decisions were affected by the instrument--the number of kindergartens in an area. Given the number of private kindergartens then, we may imagine a large portion of the population who attended kindergarten would have attended kindergarten even if they had to go to another sub-district; they may have the resources to do so. We can imagine this same population is likelier to complete high school regardless of kindergarten attendance. On the other hand, the portion of the population who would only be attending kindergarten if there were a large number in \textit{their} sub-district would be less wealthy and thus, perhaps, kindergarten would have a greater effect for this subsample. IV, by design, may be more swayed by this latter group, and thus result in higher estimates of the effect of kindergarten. This gap between estimates merits further closer investigation. This hypothesis would apply to other schooling variables as well.

\subsection{School Completion}

Completed years of education is a long-term outcome, and I can break it up into the completion of the three levels of schooling: elementary, junior high, and senior high school.\footnote{In Indonesia, elementary school ends after the completion of the 6th grade, junior high after the completion of the 9th grade, and senior high after the completion of the 12th grade.} In this section I analyze the effects of kindergartens where dummy variables indicating whether a child completed each of these levels of schooling are the outcome variables—allowing me to analyze heterogeneity in kindergarten’s effects over the life cycle and examine whether there is fade-out. Results of this analysis can be found in Table ~\ref{table:full_complete}.
\begin{center}
	[ Table ~\ref{table:full_complete} here ]
\end{center}
School completion is a slightly challenging outcome to analyze. First, for elementary school completion there is almost no variance--only 5.29\% of the sample failed to complete elementary school. I suspect in large part because of this lack of variance in elementary school completion in the sample, the OLS model finds no significant effects of kindergarten on the completion of elementary school. On the other hand, 18.68\% failed to complete junior high school and 38.19\% failed to complete senior high school.

There is significant heterogeneity across the three model types: OLS, mother fixed-effects, and IV estimation. Focusing on junior and senior high school, the OLS model finds significant positive effects for junior and senior high completion. Similar to the estimates for years of education, fixed-effects once again finds no significant effect of kindergarten attendance. The IV estimates, on the other hand, suggest fadeout; there is a very strong significantly positive effect of kindergarten attendance on the completion of junior high school that `fades' to insignificance for senior high school. We can focus on even more granular outcomes than school completion by focusing on specific grades and their rates of completion.

Importantly, the estimates for the covariates' coefficients are very similar across the OLS and IV estimates. Household expenditures become increasingly important over the life cycle (particularly the most recent observation in 2007), and the mother's years of education becomes more significant over time as well. This stands in contrast to the declining effects of kindergarten post-junior high found above, and that will also be found below in Figure ~\ref{fig:inschl_reg}, which focuses on school attendance--the next outcome I will analyze.\footnote{This difference between junior and senior high suggests that the larger effect of kindergarten on junior high is not merely a function of greater variance in the outcome variable, since we do not see a similarly increased estimate for senior high completion.}

\subsection{School Attendance and Stay-On Decision}
The next, more granular outcome to examine is grade-by-grade school attendance--captured by a dummy indicating whether a child completed every single grade.\footnote{ I focus on grades 1-14, for reasons described above.} In Figure ~\ref{fig:inschl_reg}, I have plotted the attendance rates themselves by grade and then the effects of attending kindergarten by grade. The vertical lines on the attendance rate graph indicate the 6th, 9th, and 12th grades, which are the final grades for elementary, junior high, and senior high school, respectively--note the steep drop-offs in attendance following the conclusion of each grade.
\begin{center}
	[ Figure ~\ref{fig:inschl_reg} here ]
\end{center}
We can make a few observations: first, as I mentioned above, there is nearly no variation in school attendance throughout primary school--particularly for children who attend kindergarten. However, the estimated marginal significance of kindergarten's effects throughout grades 1-6 (consisting of elementary school) suggests the gap between kindergarten and non-kindergarten children is explained by other factors. The estimated effect of kindergarten is greatest in junior high school--for the OLS and IV estimates. After junior high, however, the effects of kindergarten fade to marginal significance once again and become nearly exactly zero after high school. We once again have preliminary evidence of fadeout.

Examining school attendance over time has a critical advantage--the sample is kept constant across time. The disadvantage of this approach, however, is that once a child stops attending school, the only possible values for the remaining attendance dummy variables is 0. Thus, I corroborate analyzing school attendance with looking at `stay-on'; a dummy variable conditional on a student already being in a grade, taking on a value of 1 if they \textit{continue} in school, and 0 if they exit school after that grade.\footnote{For the stay-on analysis, the sample varies across grades since if a student had dropped out after the 4th grade, say, their assigned value for the 6th grade stay-on variable is missing and not zero. For school attendance by grade, on the other hand, the attendance dummy for 6th grade in this situation would be zero, not missing.} The results of that analysis can be found in Table ~\ref{table:full_stayon}.
\begin{center}
	[ Table ~\ref{table:full_stayon} here ]
\end{center}
These results provide strong evidence of fade-out. The OLS estimates suggest that kindergarten has a strongly significant positive effect on whether a student stays on after the 6th grade, and a weaker effect corresponding to the 9th and 12th grades. The fixed-effects model suggests there is no effect, though there is some very weak suggestion of a negative effect for whether a student stays on after the 12th grade. 

Finally, the IV estimates--my primary empirical method--show a very strong significant positive effect for 6th grade and absolutely insignificant effects for staying on after 9th and 12th grades. This corroborates the evidence from the school attendance analysis: kindergarten may not have an effect on completing elementary school, but it may lead children to attend junior high school, after which kindergarten has no \textit{enduring} effect. Interestingly, there is no evidence that the number of schools in a locality has any effect on whether a student stays on--suggesting that supply side forces are not powerful in explaining educational attainment, though this is outside the scope of this research and motivates further work.

\subsection{Cognitive Test Scores}

Thus far, my educational outcomes have been different measures of educational attainment; however, as I make clear in Section ~\ref{sec:intro}, in Indonesia, student \textit{learning} is not keeping up with rapid advances in school attendance. I am not able to match standardized test scores to individuals in the dataset; however, the IFLS has a book in which interviewees are subject to a basic cognitive test. This cognitive test has two versions: one for children aged 7-14 and the other for respondents aged 15-24. I standardize each individual's test score according to their age group's performance on the test to ensure I am only comparing like-aged individuals. Each test consists of two parts: the first is basic visual recognition of shapes and patterns. An example page of this test from the 15-24 version in 2000 can be seen in Figure ~\ref{fig:ek_shapes}.
\begin{center}
	[ Figure ~\ref{fig:ek_shapes} here ]
\end{center}
The second portion consists of basic arithmetic problems. For the 7-14 aged children, these are just math problems while for the 15-24 age group they are more involved word problems. An example page of this test from the 7-14 test in 2000 can be seen in Figure ~\ref{fig:ek_math}.\footnote{These tests are not perfect and there may be measurement error; however, all covariates except for kindergarten attendance (in particular, the mother's years of education and household expenditure over time) behave very similar for the models with schooling outcomes as for this model with cognitive test performance outcomes, so there is not something systematically biasing results across covariates/explanatory variables.}
\begin{center}
	[ Figure ~\ref{fig:ek_math} here ]
\end{center}
I then use individuals' performance on these tests in 2000, 2007, and 2014 to effectively track their cognitive abilities over time.\footnote{I create a sub-sample for this analysis, restricting the sub-sample to only those individuals who completed all 3 tests in order to effectively compare tests over-time.} The results of that analysis can be found in Table ~\ref{table:full_ek}.\footnote{All individuals completed the same cognitive test; therefore, I have added age fixed-effects to account for the differing intercept based on age that we would expect.}
\begin{center}
	[ Table ~\ref{table:full_ek} here ]
\end{center}
These results are mixed; the OLS estimates of the coefficient of kindergarten attendance tell a clear story of fade-out. Kindergarten has a strong positive effect on the 2000 scores, then a still significant and positive, although weaker, effect on 2007 scores, before fading to insignificance for the 2014 scores. The fixed-effects results suggest insignificant effects, and the IV--most perplexingly--also suggest insignificant effects. 

The IV estimates tell two very different stories of kindergarten’s effect: one for cognitive performance, where kindergarten has no significant effect, and one for educational attainment, where kindergarten has a very significant positive effect. This divergence would confirm fears outlined in Section ~\ref{sec:intro} that increasing enrollment and school completion may not be translating to better learning and effective human capital formation for students. 

We can also look at change in cognitive performance across various years: perhaps kindergarten resulted in increased performance from 2000 to 2007, even if performance remained low. First, we can look at percent change in test performance from 2000 to 2007. For both un-standardized and standardized percent change (standardized according to age), kindergarten has a positive insignificant effect. Second, looking at change from 2007 to 2014, kindergarten’s effect’s decreases while remaining insignificantly positive, for both standardized and un-standardized change. It is tempting to suggest these results suggest fadeout; however, there was no significant effect of kindergarten to begin with, preventing me from interpreting these results with any significance or weight. 

The lack of significant results using IV estimation is made more credible because the IV estimates were so large and statistically significant for the educational attainment variables. If there was some upward bias of the IV method in estimating kindergarten effects, then they would also be at work here--but we see no significant estimate of the effect of kindergarten. This divergence is also critical because the covariates previously demonstrated to be significant and critical to explaining educational attainment--namely household expenditure in 2007 and the mother's years of education--are still very significantly positive in explaining cognitive test performance. I will discuss the interpretation and significance of these results in greater detail in Section ~\ref{sec:conclusion}. 


 