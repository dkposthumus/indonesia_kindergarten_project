As discussed in the Section ~\ref{sec:lit_rev}, there are significant empirical challenges to isolating the causal effect of early childhood investments on later-life outcomes. One common challenge is selection bias, particularly in Indonesia, where kindergarten is not compulsory and in the 1990s kindergartens were almost entirely private.\footnote{See Figure ~\ref{fig:kinder_numscatter}. Based on the PODES data, in 1990, 95.06\% of kindergartens were private. That share grew to 96.63\% in 2000.} Clearly, kindergarten included a significant opportunity cost and required a significant investment of resources by a family to pay tuition and physically transport their young child to a classroom when it’s not required by law. To meet this challenge, I employ three empirical strategies: ordinary least squares (OLS) estimation, mother fixed-effects (FE), and then instrumental variable (IV) estimation.\footnote{I also use Logit regressions to support my primary analysis, such as to examine selection into attrition and kindergarten; for a description of that method, see Appendix ~\ref{app:logit}.} I believe that selection bias into kindergarten affects the OLS analysis through omitted variable bias--which motivates my use of mother fixed-effects and IV to counter.

\subsection{OLS}
My starting point is a basic OLS estimation of the effects of kindergarten, using the following model:
\begin{gather}
	Y_{if} = \beta_0 + \beta_1\text{KINDER}_{if} + \beta_2\mathbf{X}_{if} + \beta_3\mathbf{Z}_f + \beta_4\mathbf{K}_f + \epsilon_i
\end{gather}
where $Y_{if}$ is the outcome variable for individual $i$ in family $f$, $\text{KINDER}_{if}$ is a dummy variable for kindergarten attendance for individual $i$, $\mathbf{X}_{if}$ is a vector of individual-level variables, $\mathbf{Z}_f$ is a vector of household-level controls, and $\mathbf{K}_f$ is a vector of community-level controls. Our coefficient of interest is $\beta_1$, which represents the effect of kindergarten attendance on the educational outcome.

Critical to my OLS approach is that I control for the two forces driving household human capital decision-making: 1) household expenditure/resources and 2) the importance placed on education/human capital within the household.\footnote{Examples of the former include the natural log of household per-capita expenditure for every available year, tracking household expenditure across the educational career of an individual (this exogenous variable ostensibly absorbs regional economic shocks as well) and whether a household has electricity, also captured over time. Examples of the latter include a mother's or household head's years of completed education, whether an older sibling attended kindergarten, and the number of visits to a doctor by each child. These controls are implemented in conjunction with more basic individual and community controls. For example, in my analyses I control for the number of times a mother has taken an individual child to the doctor in the last few months. This, when weighed against evaluations of the health of the child, may be a proxy for that mother's willingness or desire to invest in the human capital of the child -- after all, ensuring health in early childhood has been demonstrated to greatly enhance human capital accumulation \citep{Attanasio2020}.}

As part of my OLS estimations, I also employ province fixed-effects--as it is clear that there is a heterogenous component of the relationship between kindergarten and education across provinces, as demonstrated in Figure ~\ref{fig:kinder_prov}.\footnote{This province panel variable is more general than the mother identification variable; therefore, while this fixed-effects is less powerful than mother fixed-effects, it also does not restrict my sample--as Figure ~\ref{fig:scatter_switching} makes clear, there is sufficient variation in every province so that my `switching sample' (see below) is the same as my total sample).} Finally, I use cluster-robust standard errors for the OLS estimations; thus, I cluster standard errors at the kecamatan--the equivalent of a county in Indonesia--level, making my estimates heteroskedastic-robust. As I discuss in Section ~\ref{sec:results} and Appendix ~\ref{app:ols_post}, I use a variety of post-estimation analysis to ensure my results are robust and the assumptions of OLS hold up under scrutiny.
\begin{center}
	[ Figure ~\ref{fig:kinder_prov} here ]
\end{center}

\subsection{Mother Fixed-Effects}
	\label{sec:fe}
For as many variables I can control in the above OLS model, there are critical household variables that I simply do not have data for--in particular, related to unobservable characteristics such as the emphasis placed on education within a household. Therefore, there will always be omitted variables correlated with \textit{both} kindergarten attendance and the educational outcome variables in the OLS model. I hypothesize this would result in a upwards bias in the estimate of the coefficient for kindergarten attendance, as powerful positive explanatory variables for educational outcomes correlated with kindergarten attendance are omitted: thus, by absorbing these omitted variables into the fixed-effects term, I expect non-biased estimates using the mother fixed-effects model.

This omitted variable bias motivates an oft-employed approach to examining the effects of early childhood interventions: family fixed-effects \citep{Currie1993,Garces2000}.\footnote{The stronger instance of family fixed-effects is the twin studies I described in the Section ~\ref{sec:lit_rev}.} In particular, I employed mother fixed-effects, an approach that is powerful because it immediately controls for \textit{all} household-level variables. FE accomplishes this feat by comparing children of the same mother. The fixed-effects approach estimates the following equation:
\begin{gather}
	Y_{if} = \beta_0 + \beta_1\text{KINDER}_{if} + \beta_2\mathbf{X}_{if} + \beta_3\mathbf{M}_{ft} + \mu_f + \epsilon_i
\end{gather}
Note that the variables constant within the family from equation (1) are replaced with a single fixed effects term, $\mu_f$, which captures \textit{all} mother characteristics--including all those that were not observed as part of the IFLS. Fixed-effects, however, is not a panacea; it only controls for all \textit{time-invariant} characteristics of the mother (rooted in 1997--the initial point of my sample). Therefore, I need to incorporate a vector of time-varying mother/household characteristics, represented by the term $\mathbf{M}_{ft}$.\footnote{These characteristics include economic or health shocks to the mother/household, the household economic status, the household structure (i.e., whether it remains a two-parent household or its size), and community-level shocks such as natural disasters.} It is possible that in reality a mother could treat their children differently—accordingly I’ve attempted to control for some pre-kindergarten characteristics that may signal favoritism, focusing on such indicators identified by previous work \citep{Garces2000}. 

To avoid the bias of unobserved mother favoritism, I include a variable representing the number of times a parent took a child to the hospital as well as the general health of the child in 1997. This is based on a hypothesis that parents concentrate investment into the child they believe has the best chance of success--so the more often a parent takes a child to the doctor, health held constant, the likelier they may also be to send that child to kindergarten \citep{Duflo2011}. 

It is also possible that there are spillover effects from one child attending kindergarten on successive children--something I attempt to control for by including a dummy variable for whether an older sibling had already attended kindergarten. Lastly, it is possible that a family has a different quantity of resources at its disposal when one child is of kindergarten age than when another child is--something I control for by employing controls for time-varying household characteristics captured by household expenditures. The latter is a particular concern, as fixed-effects controls for all unobserved \textit{time-invariant} characteristics. I alleviate this concern by controlling for age.

Fixed-effects introduces another complication; in order to run a regression for a specific family $f$, there must be variation in the explanatory variable, introducing a two-step requirement to be included in the fixed-effects model specification. These requirements are that 1) each mother has more than one child in our sample and 2) there is variation between a mother's children in kindergarten attendance. This significantly restricts our sample--as seen in the diagnostic graph Figure ~\ref{fig:scatter_switching}.\footnote{17.48\% of the sample are from households with exactly one child. Only 7\% of my total sample qualifies as part of a switching household according to these requirements--meaning there is a sample loss of 93\% just for the fixed-effects sample specification.}
\begin{center}
	[ Figure ~\ref{fig:scatter_switching} here ]
\end{center}
These requirements for my fixed-effects specifications thus distinguish between two types of families already included in my sample: 1) ``switching" families and 2) ``non-switching families". \citep{Miller2023} ``Switching" families are included in the fixed-effects model and ``non-switching" families are not. Often scholars are concerned about selection bias into the ``switching" designation. I find the ``switching" sample has some puzzling characteristics when compared to the total sample. There is no significant positive selection bias, as the only significant predictor of whether a family is a ``switcher" is household size, which I do not find to be significant in explaining educational outcomes. However, selection into kindergarten attendance appears to be a drastically different process--something I discuss in greater detail in Appendix ~\ref{app:kinder_sel} and Appendix ~\ref{app:switching}. 

In brief, while household characteristics function similarly in determining kindergarten attendance for switching and non-switching families, individual pre-treatment characteristics--the only category of covariates I can control for in fixed-effects--appear to function in the opposite direction. For example, for non-switching families there is a significant negative effect of whether an older sibling attended kindergarten on selection into kindergarten. Additionally, the switching sample is so small as to render any OLS or IV estimates of the effect of kindergarten restricted to the switching sub-sample insignificant. 

\subsection{Instrumental Variable (IV) Estimation}
Each of my first two empirical approaches--OLS and fixed-effects--have shortcomings. First, OLS merely suggests an association between kindergarten attendance and educational outcomes. Critically, it suffers from omitted variable bias because of a host of unobservable variables. Second, as powerful as fixed-effects is for controlling household factors, it introduces significant sample restriction and selection bias into the switching sample.

This motivates me to use instrumental variable (IV) estimation methods to remedy the bias introduced by the endogenous regressor, kindergarten attendance. I develop two instruments, both taken from the Village Potential Statistics (PODES) waves of 1990 and 2000; PODES contains explicit data about the presence of kindergartens in villages--see Section ~\ref{sec:podes} for more information. I aggregate population and kindergarten across each kecamatan and then merge based on the kecamatan an individual or household is registered under in the 1997 wave of the IFLS, creating a population-weighted average of kindergartens for each kecamatan. My first instrument consists of total (private + public) kindergartens per 10,000 people in each kecamatan in 1990 and my second instrument is the same measure in 2000.\footnote{In Appendix ~\ref{app:iv_robust} I explore alternative instruments, such as comparing just private to just public kindergartens, taking an average of the 1990 and 2000 measures, and percent change in the presence of kindergartens.} While I use the total number of kindergartens, the data clearly reflects it is private kindergartens driving expansion in the total number of programs between 1990 and 2000, as seen in Figure ~\ref{fig:kinder_numscatter}.
\begin{center}
	[ Figure ~\ref{fig:kinder_numscatter} here ]
\end{center}
There are different estimators for IV estimation; I employ the Generalized Method of Moments (GMM) estimator. In cases of over-identification, GMM does not reduce multiple instruments into one matrix and therefore is a more efficient estimator. It also gives me greater flexibility in conducting over-identification tests. The employment of an instrument requires a robust defense--both intuitively and empirically. An instrument needs to be predictive of kindergarten attendance while having no impact on educational attainment when relevant factors are controlled for \citep{Levitt2002}.

	\subsubsection{Instrument Strength} 
First, intuitively the two instruments will each be very strong: we would expect that the number of kindergartens in a locality, all else being equal, would have an effect on whether a random student from that locality attends kindergarten. This theoretical expectation is backed up empirically by the results of the first-stage regression, found in Table ~\ref{table:iv_first_stage}. One of the advantages of GMM is that it employs the two instruments separately; therefore, to demonstrate instrument strength, I regress each instrument on kindergarten attendance separately. While it does appear that the number of kindergartens per 10,000 people in 1990 is a stronger instrument, clearly the corresponding 2000 variable is significantly strong.\footnote{I use a basic F-Test to reject the null hypothesis that either coefficient is equal to 0 in the fully-specified model, each with a p-value of 0.0000. More broadly, the `rule of thumb' for first-stage regressions is to look for an F-Statistic greater than 10; for the 1990 instrument, the F-Statistic is 136.48 in the fully-specified model, and for the 2000 instrument, it is 121.78.} Since the regressor being complemented by the instrument and the instrument are the same across models--only the outcome variable changing--I only need to check instrument strength once.
\begin{center}
	[ Table ~\ref{table:iv_first_stage} here ]
\end{center}
	\subsubsection{Instrument Validity}
Second, instrument validity requires a more finely-tuned theoretical proof, as well as more detailed empirical evidence. I suspect that these instruments do not exert an effect on educational outcomes \textit{when} other community factors are controlled for. In particular, I control for the number of elementary schools per 10,000 people in each locality in 2000, junior high schools in 2007, and senior high schools in 2014; controlling for the presence of these schools, I theorize, isolates the effect of the supply of kindergartens to only kindergarten attendance. Thus, any effect that the supply of kindergartens has would be mediated through kindergarten attendance. This broad theoretical argument would apply for any of the outcome variables I research in this paper.\footnote{Not all kindergartens are the same, and I naturally expect some kindergarten programs to be of a higher quality than others. Neither PODES nor IFLS has data on the \textit{details} of kindergarten programs and their quality, so this would be an area for further research, and one I discuss in Section ~\ref{sec:conclusion}.} 

For the empirical evidence of validity, I rely on Hansen's J Test, or the over-identifying restrictions test. Two things allow me to run this test: 1) I have two instruments for one endogenous regressor and 2) I employ the GMM estimation method. I then run the over-identifying restrictions test for every model, as the outcome variables alter the calculation of the test's F-Statistic. Overall, the results of this test are promising: I fail to reject the null hypothesis that at least one of the two instruments is not valid for all models except elementary school completion, junior high school completion, school attendance for grades 3-9, and stay-on decision making after grade 6.\footnote{Elementary school completion, and school attendance for grades 3-6 do not concern me very much, as there was not enough variance in the outcome to begin with and model results are thus not very significant.} This largely corroborates the theoretical argument I outlined above.



