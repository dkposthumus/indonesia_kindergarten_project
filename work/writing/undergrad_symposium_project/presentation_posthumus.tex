\def \data #1{/Users/danielposthumus/thesis_independent_study/work/writing/rough_draft/data/#1}
\def \analysis #1{/Users/danielposthumus/thesis_independent_study/work/writing/rough_draft/analysis/#1}

\documentclass{beamer}
\setbeamertemplate{caption}[numbered]

\newcommand\Fontvi{\fontsize{9}{10}\selectfont}

\title{Starting Early: Returns on Kindergarten Attendance in Indonesia}
\author{Daniel Posthumus \\ Advisor: Ranjan Shrestha}
\institute{Economics, College of William and Mary}
\date{March 22, 2024}

\usepackage{graphicx}
\def\bibfname{pres_bib.bib}
\makeatother
\usepackage{booktabs}
\usepackage{adjustbox}
\usepackage[flushleft]{threeparttable}

\begin{document}

\frame{\titlepage}

\begin{frame}
\frametitle{Introduction}
\Fontvi
\begin{itemize}
	\item Despite rapid economic growth, quality of education has lagged behind in Indonesia
		\begin{itemize}
		\Fontvi
			\item Indonesia averaged 5.26\% economic growth from 2000 to 2019
			\item Between 2012 and 2022, boys performed worse in math, and girls didn't improve
			\item Indonesia ranked 71st in reading, 70th in math, and 67th in science (out of 81 countries) in 2022 
		\end{itemize}
	\item There's some correlation between kindergarten attendance and educational attainment:
\end{itemize}
\begin{figure}
\begin{center}
	\caption{\Fontvi Years of Education over Year of Birth, by Kindergarten Attendance}
		\includegraphics[width=2in]{\data{kinder_binscatter.png}}
\end{center}
\end{figure}
\end{frame}

\begin{frame}
\frametitle{Data}
\Fontvi
\textbf{Indonesian Family Life Survey (IFLS)}
\begin{itemize}
	\item Multi-wave household and community survey,  five waves from 1993 to 2014
	\item Tracks individuals from when they're kindergarten age in 1997 to post-graduation and adult life in 2014
\end{itemize}
\vspace{0.1in}
\textbf{Village Potential Statistics (PODES)}
\begin{itemize}
	\item Survey of 65,000 villages in Indonesia
	\item Contains data IFLS doesn't; I use the 1990 and 2000 waves to create my instrument relating to the presence of kindergartens in each kecamatan
\end{itemize}
\vspace{0.1in}
\textbf{Sample}
\begin{itemize}
	\item All individuals aged between 3 and 9 in 1997 and were interviewed in both 1997 and 2014
	\item My sample size is 3,232 individuals
\end{itemize}
\end{frame}

\begin{frame}
\frametitle{Empirical Methods}
\Fontvi
\vspace{0.1in}
\textbf{Mother fixed-effects model:}
\begin{gather}
Y_{if} = \beta_0 + \beta_1\text{KINDER}_{if} + \beta_2 \mathbf{K}_{if} + \mu_f + \epsilon_{if}
\end{gather}
$Y$ is my outcome variable, KINDER is whether a child attended kindergarten, and $\mathbf{K}$ is a vector of individual characteristics.
\vspace{0.1in} \\
\textbf{2-Stage Least Squares (2SLS) model:}
\begin{gather}
\text{Main equation:   } Y_i = \alpha_0 + \rho \text{KINDER}_i + \gamma_0 \mathbf{K}_{if} + \mathbf{C}_{f} + \epsilon_{0i} \\ 
\text{First stage:   } \text{KINDER}_i = \alpha_1 + \phi Z_i + \gamma_1 \mathbf{K}_{if} + \mathbf{C}_{f} + \epsilon_{1i}
\end{gather}
$Z$ is the instrument and $\mathbf{C}$ is a vector of household characteristics. \\
\vspace{0.1in}
Instruments:
\begin{enumerate}
	\item kindergartens per 10,000 people in each kecamatan in 1990
	\item kindergartens per 10,000 people in each kecamatan in 2000
\end{enumerate}
\end{frame}

\begin{frame}
\frametitle{Fixed Effects Results}
\Fontvi
\begin{table}
\caption{Kindergarten's Effects on Various Educational Outcomes}
\begin{adjustbox}{width=1.1\textwidth,center=\textwidth}
	\begin{tabular}{llllll}
\cline{1-6}
\multicolumn{1}{c}{} &
  \multicolumn{1}{r}{educ yrs} &
  \multicolumn{1}{r}{educ yrs} &
  \multicolumn{1}{r}{elem completion} &
  \multicolumn{1}{r}{junior completion} &
  \multicolumn{1}{r}{senior completion} \\
\cline{1-6}
\multicolumn{1}{l}{Kinder} &
  \multicolumn{1}{r}{0.70***} &
  \multicolumn{1}{r}{0.34 } &
  \multicolumn{1}{r}{0.03 } &
  \multicolumn{1}{r}{0.00 } &
  \multicolumn{1}{r}{0.09* } \\
\multicolumn{1}{l}{} &
  \multicolumn{1}{r}{(0.11)} &
  \multicolumn{1}{r}{(0.34)} &
  \multicolumn{1}{r}{(0.03)} &
  \multicolumn{1}{r}{(0.03)} &
  \multicolumn{1}{r}{(0.05)} \\
\multicolumn{1}{l}{Household Controls} &
  \multicolumn{1}{r}{YES} &
  \multicolumn{1}{r}{NO} &
  \multicolumn{1}{r}{NO} &
  \multicolumn{1}{r}{NO} &
  \multicolumn{1}{r}{NO} \\
\multicolumn{1}{l}{Individual Controls} &
  \multicolumn{1}{r}{YES} &
  \multicolumn{1}{r}{YES} &
  \multicolumn{1}{r}{YES} &
  \multicolumn{1}{r}{YES} &
  \multicolumn{1}{r}{YES} \\
\multicolumn{1}{l}{Mother Fixed-Effects} &
  \multicolumn{1}{r}{NO} &
  \multicolumn{1}{r}{YES} &
  \multicolumn{1}{r}{YES} &
  \multicolumn{1}{r}{YES} &
  \multicolumn{1}{r}{YES} \\
\multicolumn{1}{l}{Adjusted R-squared} &
  \multicolumn{1}{r}{0.34} &
  \multicolumn{1}{r}{0.01} &
  \multicolumn{1}{r}{0.01} &
  \multicolumn{1}{r}{0.01} &
  \multicolumn{1}{r}{0.01} \\
\multicolumn{1}{l}{Number of observations} &
  \multicolumn{1}{r}{3232} &
  \multicolumn{1}{r}{3233} &
  \multicolumn{1}{r}{3233} &
  \multicolumn{1}{r}{3233} &
  \multicolumn{1}{r}{3233} \\
\cline{1-6}
\end{tabular}

\end{adjustbox}
\label{tab:fe_results}
\end{table}
\end{frame}

\begin{frame}
\frametitle{Instrumental Variable Estimation Results}
\Fontvi
\begin{table}
\caption{Instrumental Variable (IV) Estimation Results}
\begin{adjustbox}{width=0.8\textwidth,center=\textwidth}
	\input{\analysis{iv_results_main.tex}}
\end{adjustbox}
\end{table}

\begin{table}
\caption{First Stage Regression Results}
\begin{adjustbox}{width=0.8\textwidth,center=\textwidth}
	\input{\analysis{iv_first_stage.tex}}
\end{adjustbox}
\end{table}
\end{frame}

\end{document}