\def \data #1{/Users/danielposthumus/thesis_independent_study/work/writing/rough_draft/data/#1}
\def \analysis #1{/Users/danielposthumus/thesis_independent_study/work/writing/rough_draft/analysis/#1}

\documentclass{beamer}
\setbeamertemplate{caption}[numbered]

\newcommand\Fontvi{\fontsize{9}{10}\selectfont}

\title{Starting Early: Returns on Kindergarten Attendance in Indonesia}
\author{Daniel Posthumus \\ Advisor: Ranjan Shrestha}
\institute{Department of Economics \\ College of William and Mary}
\date{May 02, 2024}

\usepackage{graphicx}
\def\bibfname{pres_bib.bib}
\makeatother
\usepackage{booktabs,caption}
\usepackage{adjustbox}
\usepackage[flushleft,para]{threeparttable}
\usepackage{comment}
\usepackage{changepage}
\usepackage[style=authoryear]{biblatex}
	\addbibresource{bibliography.bib}

\begin{document}

\frame{\titlepage}

\begin{frame}
\frametitle{Introduction}
\begin{itemize}
	\item Rapid economic growth in Indonesia:
	\begin{itemize} 
		\item Averaging 5.26\% economic growth from 2000 to 2019
	\end{itemize}
	\vspace{0.1in}
	\item Robust post-COVID recovery
	\vspace{0.1in}
	\item Growth has been accompanied by significant investments in education:
	\begin{itemize}
		\item 200\% real increase in education spending from 2002 to 2018 
	\end{itemize}
	\vspace{0.1in}
	\item Rapid gains in schooling:
	\begin{itemize}
		\item Completion of primary school is now near-universal
		\item Gender parity in schooling
	\end{itemize}
\end{itemize}
\end{frame}

\begin{frame}
\frametitle{Introduction}
\begin{itemize}
	\item Human capital has not kept up with growth:
	\begin{itemize}
		\item 87th in the world in Human Capital Index
	\end{itemize}
	\vspace{0.1in}
	\item While schooling has improved, learning has not
	\begin{itemize}
		\item Boys' test scores worsened in math, and girls' didn't improve (2012-2022)
		\item 71st in reading, 70th in math, and 67th in science (out of 81 countries)
		\item Students lost 11 months of school due to COVID
	\end{itemize}
\end{itemize}
\end{frame}

\begin{frame}
\frametitle{Motivation to Study Kindergarten}
\begin{itemize}	
	\item Possible solution to improve human capital: \textbf{quality early childhood education}
	\vspace{0.1in}
	\item Kindergarten is currently sparse in Indonesia, mostly private
	\vspace{0.1in}
	\item Little understanding of the educational effects of kindergarten in Indonesia
\end{itemize}
\vspace{0.2in}
Research question: \\
\textbf{What are kindergarten's effects on educational outcomes?}
\end{frame}

\begin{frame}
\frametitle{Kindergarten's Positive Association with Schooling}
\begin{figure}
\begin{center}
\caption{Years of Education and Year of Birth, by Kindergarten Attendance}
\vspace{0.25in}
		\includegraphics[width=3in]{\data{kinder_binscatter.png}}
\end{center}
\end{figure}
\end{frame}

\begin{frame}
\frametitle{Empirical Findings on Kindergarten's Effects}
\begin{itemize}
	\item \textbf{Early childhood interventions are effective}, with considerable heterogeneity \parencite{Garces2000}
	\vspace{0.1in}
	\item Effects of programs may \textbf{`fade out'} as children age \parencite{Abenavoli2019} 
	\vspace{0.1in}
	\item Work on preschool in developing countries is limited
	\vspace{0.1in}
	\item Some evidence positive preschool effects translate to developing countries \parencite{Behrman2013}
\end{itemize}
\vspace{0.1in}
\textbf{No previous examination of kindergarten's effects over the life-cycle in Indonesia}
\end{frame}

\begin{frame}
\frametitle{Data}
\textbf{Indonesian Family Life Survey (IFLS)}
\begin{itemize}
	\item Multi-wave household and community survey:
	\begin{itemize}
		\item \textbf{Five waves from 1993 to 2014}
		\item Initial 1993 wave was representative of 83\% of the total population
	\end{itemize}
	\vspace{0.1in}
	\item Tracks individuals from pre-kindergarten to adulthood
\end{itemize}
\vspace{0.2in}
\textbf{Village Potential Statistics (PODES)}
\begin{itemize}
	\item Survey of \textbf{65,000 villages}
	\vspace{0.1in}
	\item Contains critical data IFLS does not:
	\begin{itemize}
		\item 1990 and 2000 data on \textbf{number of kindergartens and population} in each \textit{kecamatan}, i.e. sub-district
	\end{itemize}
\end{itemize}
\end{frame}

\begin{frame}
\frametitle{Sample}
\begin{itemize}
	\item All individuals who: 
	\begin{itemize}
		\item Were between 3 and 9 years old in 1997
		\item Individually interviewed in both 1997 and 2014
	\end{itemize}
	\vspace{0.1in}
	\item Main sample: \textbf{3,158 individuals}
	\begin{itemize} 
		\item `Switcher' sample: 221 individuals
	\end{itemize}
	\vspace{0.1in}
	\item Non-random attrition:
	\begin{itemize}
		\item Weighting by attrition likelihood does \textit{not} alter results
	\end{itemize}
	\vspace{0.1in}
	\item Sample approximates educational characteristics of general population
\end{itemize}
\end{frame}

\begin{frame}
\frametitle{Sample Summary Statistics -- Outcome Variables}
\begin{table}
	\caption{Mean and Standard Deviation of Educational Outcome Variables and Kindergarten Attendance}
	\begin{threeparttable}
		\tiny{\begin{tabular}{llllll}
\cline{1-6}
\multicolumn{1}{c}{} &
  \multicolumn{1}{r}{Full Sample} &
  \multicolumn{2}{c}{Urban} &
  \multicolumn{2}{c}{Rural} \\ \cmidrule(lr){2-2} \cmidrule(lr){3-4} \cmidrule(lr){5-6}
\multicolumn{1}{c}{} &
  \multicolumn{1}{r}{} &
  \multicolumn{1}{c}{Kinder} &
  \multicolumn{1}{c}{No Kinder} &
  \multicolumn{1}{c}{Kinder} &
  \multicolumn{1}{c}{No Kinder} \\
\cline{1-6}
\multicolumn{1}{l}{Kindergarten attendance} &
  \multicolumn{1}{c}{0.39} &
  \multicolumn{1}{c}{1.00} &
  \multicolumn{1}{c}{0.00} &
  \multicolumn{1}{c}{1.00} &
  \multicolumn{1}{c}{0.00} \\
\multicolumn{1}{l}{} &
  \multicolumn{1}{c}{(0.49)} &
  \multicolumn{1}{c}{(0.00)} &
  \multicolumn{1}{c}{(0.00)} &
  \multicolumn{1}{c}{(0.00)} &
  \multicolumn{1}{c}{(0.00)} \\
\multicolumn{1}{l}{Years of education} &
  \multicolumn{1}{c}{10.97} &
  \multicolumn{1}{c}{12.79} &
  \multicolumn{1}{c}{10.79} &
  \multicolumn{1}{c}{11.92} &
  \multicolumn{1}{c}{9.74} \\
\multicolumn{1}{l}{} &
  \multicolumn{1}{c}{(3.45)} &
  \multicolumn{1}{c}{(2.53)} &
  \multicolumn{1}{c}{(3.04)} &
  \multicolumn{1}{c}{(2.82)} &
  \multicolumn{1}{c}{(3.73)} \\
\multicolumn{1}{l}{Completed elementary} &
  \multicolumn{1}{c}{0.95} &
  \multicolumn{1}{c}{0.99} &
  \multicolumn{1}{c}{0.96} &
  \multicolumn{1}{c}{0.99} &
  \multicolumn{1}{c}{0.90} \\
\multicolumn{1}{l}{} &
  \multicolumn{1}{c}{(0.22)} &
  \multicolumn{1}{c}{(0.08)} &
  \multicolumn{1}{c}{(0.20)} &
  \multicolumn{1}{c}{(0.11)} &
  \multicolumn{1}{c}{(0.30)} \\
\multicolumn{1}{l}{Completed junior high} &
  \multicolumn{1}{c}{0.81} &
  \multicolumn{1}{c}{0.95} &
  \multicolumn{1}{c}{0.82} &
  \multicolumn{1}{c}{0.92} &
  \multicolumn{1}{c}{0.69} \\
\multicolumn{1}{l}{} &
  \multicolumn{1}{c}{(0.39)} &
  \multicolumn{1}{c}{(0.21)} &
  \multicolumn{1}{c}{(0.39)} &
  \multicolumn{1}{c}{(0.27)} &
  \multicolumn{1}{c}{(0.46)} \\
\multicolumn{1}{l}{Completed senior high} &
  \multicolumn{1}{c}{0.62} &
  \multicolumn{1}{c}{0.85} &
  \multicolumn{1}{c}{0.61} &
  \multicolumn{1}{c}{0.72} &
  \multicolumn{1}{c}{0.46} \\
\multicolumn{1}{l}{} &
  \multicolumn{1}{c}{(0.49)} &
  \multicolumn{1}{c}{(0.36)} &
  \multicolumn{1}{c}{(0.49)} &
  \multicolumn{1}{c}{(0.45)} &
  \multicolumn{1}{c}{(0.50)} \\
\multicolumn{1}{l}{Cognitive score, ln (2000)} &
  \multicolumn{1}{c}{-0.46} &
  \multicolumn{1}{c}{-0.33} &
  \multicolumn{1}{c}{-0.44} &
  \multicolumn{1}{c}{-0.41} &
  \multicolumn{1}{c}{-0.55} \\
\multicolumn{1}{l}{} &
  \multicolumn{1}{c}{(0.38)} &
  \multicolumn{1}{c}{(0.28)} &
  \multicolumn{1}{c}{(0.35)} &
  \multicolumn{1}{c}{(0.34)} &
  \multicolumn{1}{c}{(0.43)} \\
\multicolumn{1}{l}{Cognitive score, ln (2007)} &
  \multicolumn{1}{c}{-0.43} &
  \multicolumn{1}{c}{-0.28} &
  \multicolumn{1}{c}{-0.41} &
  \multicolumn{1}{c}{-0.37} &
  \multicolumn{1}{c}{-0.54} \\
\multicolumn{1}{l}{} &
  \multicolumn{1}{c}{(0.38)} &
  \multicolumn{1}{c}{(0.26)} &
  \multicolumn{1}{c}{(0.33)} &
  \multicolumn{1}{c}{(0.35)} &
  \multicolumn{1}{c}{(0.43)} \\
\multicolumn{1}{l}{Cognitive score, ln (2014)} &
  \multicolumn{1}{c}{-0.56} &
  \multicolumn{1}{c}{-0.42} &
  \multicolumn{1}{c}{-0.55} &
  \multicolumn{1}{c}{-0.47} &
  \multicolumn{1}{c}{-0.66} \\
\multicolumn{1}{l}{} &
  \multicolumn{1}{c}{(0.41)} &
  \multicolumn{1}{c}{(0.34)} &
  \multicolumn{1}{c}{(0.35)} &
  \multicolumn{1}{c}{(0.37)} &
  \multicolumn{1}{c}{(0.45)} \\
\multicolumn{1}{l}{Number of Observations} &
  \multicolumn{1}{c}{3158} &
  \multicolumn{1}{c}{700} &
  \multicolumn{1}{c}{571} &
  \multicolumn{1}{c}{533} &
  \multicolumn{1}{c}{1350} \\
\cline{1-6}
\end{tabular}
}
	\begin{tablenotes}
		\item {\tiny Note: Figures in parentheses are standard deviations of each variable.}
	\end{tablenotes} 
	\end{threeparttable}
\end{table}
\end{frame}

\begin{frame}
\frametitle{Sample Summary Statistics -- Covariates}
\begin{table}
	\caption{Mean and Standard Deviation of Key Household Characteristics}
	\begin{threeparttable}
		\tiny{\begin{tabular}{llllll}
\cline{1-6}
\multicolumn{1}{c}{} &
  \multicolumn{1}{r}{Full Sample} &
  \multicolumn{2}{c}{Urban} &
  \multicolumn{2}{c}{Rural} \\ \cmidrule(lr){2-2} \cmidrule(lr){3-4} \cmidrule(lr){5-6}
\multicolumn{1}{c}{} &
  \multicolumn{1}{r}{} &
  \multicolumn{1}{c}{Kinder} &
  \multicolumn{1}{c}{No Kinder} &
  \multicolumn{1}{c}{Kinder} &
  \multicolumn{1}{c}{No Kinder} \\
\cline{1-6}
\multicolumn{1}{l}{Mom's yrs of education} &
  \multicolumn{1}{c}{5.73} &
  \multicolumn{1}{c}{8.68} &
  \multicolumn{1}{c}{5.28} &
  \multicolumn{1}{c}{6.79} &
  \multicolumn{1}{c}{3.98} \\
\multicolumn{1}{l}{} &
  \multicolumn{1}{c}{(3.97)} &
  \multicolumn{1}{c}{(3.75)} &
  \multicolumn{1}{c}{(3.32)} &
  \multicolumn{1}{c}{(3.92)} &
  \multicolumn{1}{c}{(3.28)} \\
\multicolumn{1}{l}{HH head's yrs of education} &
  \multicolumn{1}{c}{6.06} &
  \multicolumn{1}{c}{8.73} &
  \multicolumn{1}{c}{6.04} &
  \multicolumn{1}{c}{6.48} &
  \multicolumn{1}{c}{4.53} \\
\multicolumn{1}{l}{} &
  \multicolumn{1}{c}{(4.26)} &
  \multicolumn{1}{c}{(4.15)} &
  \multicolumn{1}{c}{(3.83)} &
  \multicolumn{1}{c}{(4.37)} &
  \multicolumn{1}{c}{(3.69)} \\
\multicolumn{1}{l}{HH per-capita expenditure (1997)} &
  \multicolumn{1}{c}{12.13} &
  \multicolumn{1}{c}{12.42} &
  \multicolumn{1}{c}{12.11} &
  \multicolumn{1}{c}{12.29} &
  \multicolumn{1}{c}{11.94} \\
\multicolumn{1}{l}{} &
  \multicolumn{1}{c}{(0.69)} &
  \multicolumn{1}{c}{(0.74)} &
  \multicolumn{1}{c}{(0.65)} &
  \multicolumn{1}{c}{(0.71)} &
  \multicolumn{1}{c}{(0.62)} \\
\multicolumn{1}{l}{HH per-capita expenditure (2000)} &
  \multicolumn{1}{c}{12.14} &
  \multicolumn{1}{c}{12.47} &
  \multicolumn{1}{c}{12.08} &
  \multicolumn{1}{c}{12.24} &
  \multicolumn{1}{c}{11.96} \\
\multicolumn{1}{l}{} &
  \multicolumn{1}{c}{(0.65)} &
  \multicolumn{1}{c}{(0.66)} &
  \multicolumn{1}{c}{(0.62)} &
  \multicolumn{1}{c}{(0.62)} &
  \multicolumn{1}{c}{(0.59)} \\
\multicolumn{1}{l}{HH per-capita expenditure (2007)} &
  \multicolumn{1}{c}{12.93} &
  \multicolumn{1}{c}{13.27} &
  \multicolumn{1}{c}{12.92} &
  \multicolumn{1}{c}{12.94} &
  \multicolumn{1}{c}{12.76} \\
\multicolumn{1}{l}{} &
  \multicolumn{1}{c}{(0.69)} &
  \multicolumn{1}{c}{(0.68)} &
  \multicolumn{1}{c}{(0.62)} &
  \multicolumn{1}{c}{(0.70)} &
  \multicolumn{1}{c}{(0.66)} \\
\multicolumn{1}{l}{Number of children in HH} &
  \multicolumn{1}{c}{2.60} &
  \multicolumn{1}{c}{2.20} &
  \multicolumn{1}{c}{2.77} &
  \multicolumn{1}{c}{2.34} &
  \multicolumn{1}{c}{2.83} \\
\multicolumn{1}{l}{} &
  \multicolumn{1}{c}{(1.21)} &
  \multicolumn{1}{c}{(0.91)} &
  \multicolumn{1}{c}{(1.22)} &
  \multicolumn{1}{c}{(1.03)} &
  \multicolumn{1}{c}{(1.33)} \\
\multicolumn{1}{l}{Number of Observations} &
  \multicolumn{1}{c}{3158} &
  \multicolumn{1}{c}{700} &
  \multicolumn{1}{c}{571} &
  \multicolumn{1}{c}{533} &
  \multicolumn{1}{c}{1350} \\
\cline{1-6}
\end{tabular}
}
	\begin{tablenotes}
		\item {\tiny Note: Figures in parentheses are standard deviations of each variable.}
	\end{tablenotes} 
	\end{threeparttable}
\end{table}
\end{frame}

\begin{frame}
\frametitle{Instruments}
\begin{figure}
\begin{center}
\caption{Kindergartens per 10,000 People / Kecamatan, 1990 and 2000}
	\includegraphics[width=3.75in]{\data{kinder_box.png}}
\end{center}
\end{figure}
\end{frame}

\begin{frame}
\frametitle{Schools Over Time}
\begin{figure}
\begin{center}
\caption{Schools per 10,000 People / Kabupaten, Selected Years}
	\includegraphics[width=3.75in]{\data{schl_box.png}}
\end{center}
\end{figure}
\end{frame}

\begin{frame}
\frametitle{Mother Fixed-Effects}
\begin{itemize}
	\item OLS model: omitted variable bias
	\begin{itemize}
		\item Unobservable household characteristics affect educational outcomes
	\end{itemize}
	\vspace{0.1in}
	\item Fixed-effects: \textit{all} mother characteristics are controlled for
\end{itemize}
\begin{gather}
Y_{if} = \beta_0 + \beta_1\text{KINDER}_{if} + \beta_2 \mathbf{K}_{if} + \mu_f + \epsilon_{if}
\end{gather}
\vspace{-0.2in}
{\footnotesize \begin{itemize}
	\item Individual $i$ and family $f$
	\item $Y$ is outcome variable
	\item KINDER is whether a child attended kindergarten
	\item $\mu_f$ is the mother fixed-effects 
	\item $\mathbf{K}$ is a vector of individual characteristics
\end{itemize} }
\end{frame}

\begin{frame}
\frametitle{Instrumental Variable (IV) Estimation}
\begin{itemize} 
	\item IV is one way to overcome endogeneity of kindergarten attendance
\end{itemize}
\vspace{0.1in}
\begin{itemize}
	\item[] Main equation:
\end{itemize}
\vspace{-0.1in}
\begin{gather}
Y_i = \alpha_0 + \rho \hat{\text{KINDER}_i} + \gamma_0 \mathbf{K}_{if} + \beta_0 \mathbf{C}_{f} + \epsilon_{0i}
\end{gather}
\vspace{-0.2in}
\begin{itemize}
	\item[] First stage:
\end{itemize}
\vspace{-0.1in}
\begin{gather}
\text{KINDER}_i = \alpha_1 + \phi Z_f + \gamma_1 \mathbf{K}_{if} + \beta_1 \mathbf{C}_{f} + \epsilon_{1i}
\end{gather}
\vspace{-0.2in}
{\footnotesize \begin{itemize}
	\item $Z$ is the instrument
	\item $\mathbf{C}$ is a vector of household and community characteristics
\end{itemize} }
\vspace{0.1in}
\begin{itemize}
	\item[] \textbf{Instruments:}
\end{itemize}
\vspace{-0.05in}
\begin{enumerate}
	\item kindergartens per 10,000 people / kecamatan in 1990
	\item kindergartens per 10,000 people / kecamatan in 2000
\end{enumerate}
\end{frame}

\begin{frame}
\frametitle{Instrument Validity and Strength}
Instruments have to be (1) strong and (2) valid:
\vspace{0.2in}
\begin{itemize}
	\item[(1)] \textbf{Strength}: 
	\begin{itemize}
		\item Strongly correlated with kindergarten attendance
		\item Statistically significant in first-stage regression
	\end{itemize}
\vspace{0.2in}
	\item[(2)] \textbf{Validity}: 
	\begin{itemize} 
		\item Exogenous to educational outcomes when kindergarten is controlled for 
		\item Over-identifying test
	\end{itemize}
\end{itemize}
\end{frame}

\begin{frame}
\frametitle{Results}
Educational outcomes of interest:
\vspace{0.1in}
\begin{itemize}
	\item Years of Education Completed
	\vspace{0.1in}
	\item School Completion 
	\vspace{0.1in}
	\item School Attendance and Stay-On Decision 
	\vspace{0.1in}
	\item Cognitive Test Scores
\end{itemize}
\end{frame}

\begin{frame}
\frametitle{Years of Education Completed}
\begin{table}
	\caption{Kindergarten's Effects on Completed Years of Education}
	\begin{threeparttable}
		{\tiny\begin{tabular}{llll}
\cline{1-4}
\multicolumn{1}{c}{} &
  \multicolumn{1}{c}{(1)} &
  \multicolumn{1}{c}{(2)} &
  \multicolumn{1}{c}{(3)} \\
\cline{1-4}
\multicolumn{1}{l}{Kindergarten} &
  \multicolumn{1}{c}{0.74***} &
  \multicolumn{1}{c}{-0.06 } &
  \multicolumn{1}{c}{1.70***} \\
\multicolumn{1}{l}{} &
  \multicolumn{1}{c}{(0.13)} &
  \multicolumn{1}{c}{(0.44)} &
  \multicolumn{1}{c}{(0.64)} \\
\multicolumn{1}{l}{Mom's yrs of education} &
  \multicolumn{1}{c}{0.16***} &
  \multicolumn{1}{c}{} &
  \multicolumn{1}{c}{0.13***} \\
\multicolumn{1}{l}{} &
  \multicolumn{1}{c}{(0.02)} &
  \multicolumn{1}{c}{} &
  \multicolumn{1}{c}{(0.02)} \\
\multicolumn{1}{l}{HH per-capita expenditure (1997)} &
  \multicolumn{1}{c}{0.14 } &
  \multicolumn{1}{c}{} &
  \multicolumn{1}{c}{0.11 } \\
\multicolumn{1}{l}{} &
  \multicolumn{1}{c}{(0.09)} &
  \multicolumn{1}{c}{} &
  \multicolumn{1}{c}{(0.09)} \\
\multicolumn{1}{l}{HH per-capita expenditure (2000)} &
  \multicolumn{1}{c}{0.36***} &
  \multicolumn{1}{c}{} &
  \multicolumn{1}{c}{0.32***} \\
\multicolumn{1}{l}{} &
  \multicolumn{1}{c}{(0.10)} &
  \multicolumn{1}{c}{} &
  \multicolumn{1}{c}{(0.11)} \\
\multicolumn{1}{l}{HH per-capita expenditure (2007)} &
  \multicolumn{1}{c}{1.14***} &
  \multicolumn{1}{c}{} &
  \multicolumn{1}{c}{1.12***} \\
\multicolumn{1}{l}{} &
  \multicolumn{1}{c}{(0.09)} &
  \multicolumn{1}{c}{} &
  \multicolumn{1}{c}{(0.09)} \\
\multicolumn{1}{l}{Model} &
  \multicolumn{1}{c}{OLS} &
  \multicolumn{1}{c}{FE} &
  \multicolumn{1}{c}{IV} \\
\multicolumn{1}{l}{Adjusted R-squared} &
  \multicolumn{1}{c}{0.38} &
  \multicolumn{1}{c}{0.07} &
  \multicolumn{1}{c}{0.37} \\
\multicolumn{1}{l}{Number of observations} &
  \multicolumn{1}{c}{3154} &
  \multicolumn{1}{c}{221} &
  \multicolumn{1}{c}{3154} \\
\cline{1-4}
\end{tabular}
}
	\begin{tablenotes}
		\item {\tiny Heteroskedastic-robust standard errors are reported in parentheses. \\ $^{***}p<0.01$; $^{**}p<0.05$; $^{*}p<0.10$}
	\end{tablenotes} 
	\end{threeparttable}
\end{table}
\end{frame}

\begin{frame}
\frametitle{School Completion}
\begin{adjustwidth}{-2em}{-2em}
\begin{table}
	\caption{Kindergarten's Effects on Elementary/Junior/Senior High Completion}
	\begin{threeparttable}
		{\tiny\begin{tabular}{lllllll}
\cline{1-7}
\multicolumn{1}{c}{} &
  \multicolumn{2}{c}{Elementary} &
  \multicolumn{2}{c}{Junior High} &
  \multicolumn{2}{c}{Senior High} \\
\cline{1-7}
\multicolumn{1}{l}{Kindergarten} &
  \multicolumn{1}{c}{0.01* } &
  \multicolumn{1}{c}{0.10** } &
  \multicolumn{1}{c}{0.07***} &
  \multicolumn{1}{c}{0.28***} &
  \multicolumn{1}{c}{0.10***} &
  \multicolumn{1}{c}{0.12 } \\
\multicolumn{1}{l}{} &
  \multicolumn{1}{c}{(0.01)} &
  \multicolumn{1}{c}{(0.04)} &
  \multicolumn{1}{c}{(0.02)} &
  \multicolumn{1}{c}{(0.08)} &
  \multicolumn{1}{c}{(0.02)} &
  \multicolumn{1}{c}{(0.10)} \\
\multicolumn{1}{l}{Mom's yrs of education} &
  \multicolumn{1}{c}{0.00 } &
  \multicolumn{1}{c}{-0.00 } &
  \multicolumn{1}{c}{0.01***} &
  \multicolumn{1}{c}{0.00 } &
  \multicolumn{1}{c}{0.02***} &
  \multicolumn{1}{c}{0.02***} \\
\multicolumn{1}{l}{} &
  \multicolumn{1}{c}{(0.00)} &
  \multicolumn{1}{c}{(0.00)} &
  \multicolumn{1}{c}{(0.00)} &
  \multicolumn{1}{c}{(0.00)} &
  \multicolumn{1}{c}{(0.00)} &
  \multicolumn{1}{c}{(0.00)} \\
\multicolumn{1}{l}{HH per-capita expenditure (1997)} &
  \multicolumn{1}{c}{-0.00 } &
  \multicolumn{1}{c}{-0.00 } &
  \multicolumn{1}{c}{0.00 } &
  \multicolumn{1}{c}{-0.01 } &
  \multicolumn{1}{c}{0.02 } &
  \multicolumn{1}{c}{0.02 } \\
\multicolumn{1}{l}{} &
  \multicolumn{1}{c}{(0.01)} &
  \multicolumn{1}{c}{(0.01)} &
  \multicolumn{1}{c}{(0.01)} &
  \multicolumn{1}{c}{(0.01)} &
  \multicolumn{1}{c}{(0.01)} &
  \multicolumn{1}{c}{(0.01)} \\
\multicolumn{1}{l}{HH per-capita expenditure (2000)} &
  \multicolumn{1}{c}{0.01 } &
  \multicolumn{1}{c}{0.00 } &
  \multicolumn{1}{c}{0.02 } &
  \multicolumn{1}{c}{0.01 } &
  \multicolumn{1}{c}{0.04** } &
  \multicolumn{1}{c}{0.04** } \\
\multicolumn{1}{l}{} &
  \multicolumn{1}{c}{(0.01)} &
  \multicolumn{1}{c}{(0.01)} &
  \multicolumn{1}{c}{(0.01)} &
  \multicolumn{1}{c}{(0.01)} &
  \multicolumn{1}{c}{(0.02)} &
  \multicolumn{1}{c}{(0.02)} \\
\multicolumn{1}{l}{HH per-capita expenditure (2007)} &
  \multicolumn{1}{c}{0.04***} &
  \multicolumn{1}{c}{0.04***} &
  \multicolumn{1}{c}{0.08***} &
  \multicolumn{1}{c}{0.08***} &
  \multicolumn{1}{c}{0.12***} &
  \multicolumn{1}{c}{0.12***} \\
\multicolumn{1}{l}{} &
  \multicolumn{1}{c}{(0.01)} &
  \multicolumn{1}{c}{(0.01)} &
  \multicolumn{1}{c}{(0.01)} &
  \multicolumn{1}{c}{(0.01)} &
  \multicolumn{1}{c}{(0.01)} &
  \multicolumn{1}{c}{(0.01)} \\
\multicolumn{1}{l}{Model} &
  \multicolumn{1}{c}{OLS} &
  \multicolumn{1}{c}{IV} &
  \multicolumn{1}{c}{OLS} &
  \multicolumn{1}{c}{IV} &
  \multicolumn{1}{c}{OLS} &
  \multicolumn{1}{c}{IV} \\
\multicolumn{1}{l}{Adjusted R-squared} &
  \multicolumn{1}{c}{0.08} &
  \multicolumn{1}{c}{0.05} &
  \multicolumn{1}{c}{0.18} &
  \multicolumn{1}{c}{0.14} &
  \multicolumn{1}{c}{0.26} &
  \multicolumn{1}{c}{0.26} \\
\multicolumn{1}{l}{Number of observations} &
  \multicolumn{1}{c}{3154} &
  \multicolumn{1}{c}{3154} &
  \multicolumn{1}{c}{3154} &
  \multicolumn{1}{c}{3154} &
  \multicolumn{1}{c}{3154} &
  \multicolumn{1}{c}{3154} \\
\cline{1-7}
\end{tabular}
}
	\begin{tablenotes}
		\item {\tiny Heteroskedastic-robust standard errors are reported in parentheses. \\ $^{***}p<0.01$; $^{**}p<0.05$; $^{*}p<0.10$}
	\end{tablenotes} 
	\end{threeparttable}
	\end{table}
\end{adjustwidth}
\end{frame}

\begin{frame} 
\frametitle{School Attendance}
\begin{center}
\begin{figure}
\caption{Marginal Effects of Kindergarten on Attendance, by Grade}
 	\includegraphics[width=2.1in]{\data{in_school.png}}
	\includegraphics[width=2.1in]{\analysis{in_schl_reg_bas.png}}
\end{figure}
\end{center}
\end{frame}

\begin{frame} 
\frametitle{Stay-On}
\begin{adjustwidth}{-2em}{-2em}
\begin{table}
	\caption{Kindergarten's Effects on Stay-On, for Selected Grades}
	\begin{threeparttable}
 		{\tiny\begin{tabular}{lllllll}
\cline{1-7}
\multicolumn{1}{c}{} &
  \multicolumn{2}{c}{6th Grade} &
  \multicolumn{2}{c}{9th Grade} &
  \multicolumn{2}{c}{12th Grade} \\
\cline{1-7}
\multicolumn{1}{l}{Kindergarten} &
  \multicolumn{1}{c}{0.05***} &
  \multicolumn{1}{c}{0.16** } &
  \multicolumn{1}{c}{0.04** } &
  \multicolumn{1}{c}{-0.01 } &
  \multicolumn{1}{c}{0.05* } &
  \multicolumn{1}{c}{-0.03 } \\
\multicolumn{1}{l}{} &
  \multicolumn{1}{c}{(0.01)} &
  \multicolumn{1}{c}{(0.06)} &
  \multicolumn{1}{c}{(0.02)} &
  \multicolumn{1}{c}{(0.09)} &
  \multicolumn{1}{c}{(0.03)} &
  \multicolumn{1}{c}{(0.12)} \\
\multicolumn{1}{l}{Mom's yrs of education} &
  \multicolumn{1}{c}{0.01***} &
  \multicolumn{1}{c}{0.00 } &
  \multicolumn{1}{c}{0.01***} &
  \multicolumn{1}{c}{0.02***} &
  \multicolumn{1}{c}{0.02***} &
  \multicolumn{1}{c}{0.02***} \\
\multicolumn{1}{l}{} &
  \multicolumn{1}{c}{(0.00)} &
  \multicolumn{1}{c}{(0.00)} &
  \multicolumn{1}{c}{(0.00)} &
  \multicolumn{1}{c}{(0.00)} &
  \multicolumn{1}{c}{(0.00)} &
  \multicolumn{1}{c}{(0.00)} \\
\multicolumn{1}{l}{HH per-capita expenditure (1997)} &
  \multicolumn{1}{c}{-0.00 } &
  \multicolumn{1}{c}{-0.01 } &
  \multicolumn{1}{c}{0.02* } &
  \multicolumn{1}{c}{0.03** } &
  \multicolumn{1}{c}{0.02 } &
  \multicolumn{1}{c}{0.03 } \\
\multicolumn{1}{l}{} &
  \multicolumn{1}{c}{(0.01)} &
  \multicolumn{1}{c}{(0.01)} &
  \multicolumn{1}{c}{(0.01)} &
  \multicolumn{1}{c}{(0.01)} &
  \multicolumn{1}{c}{(0.02)} &
  \multicolumn{1}{c}{(0.02)} \\
\multicolumn{1}{l}{HH per-capita expenditure (2000)} &
  \multicolumn{1}{c}{0.01 } &
  \multicolumn{1}{c}{0.00 } &
  \multicolumn{1}{c}{0.03** } &
  \multicolumn{1}{c}{0.03** } &
  \multicolumn{1}{c}{0.06***} &
  \multicolumn{1}{c}{0.06***} \\
\multicolumn{1}{l}{} &
  \multicolumn{1}{c}{(0.01)} &
  \multicolumn{1}{c}{(0.01)} &
  \multicolumn{1}{c}{(0.01)} &
  \multicolumn{1}{c}{(0.02)} &
  \multicolumn{1}{c}{(0.02)} &
  \multicolumn{1}{c}{(0.02)} \\
\multicolumn{1}{l}{HH per-capita expenditure (2007)} &
  \multicolumn{1}{c}{0.04***} &
  \multicolumn{1}{c}{0.04***} &
  \multicolumn{1}{c}{0.06***} &
  \multicolumn{1}{c}{0.07***} &
  \multicolumn{1}{c}{0.15***} &
  \multicolumn{1}{c}{0.15***} \\
\multicolumn{1}{l}{} &
  \multicolumn{1}{c}{(0.01)} &
  \multicolumn{1}{c}{(0.01)} &
  \multicolumn{1}{c}{(0.01)} &
  \multicolumn{1}{c}{(0.01)} &
  \multicolumn{1}{c}{(0.02)} &
  \multicolumn{1}{c}{(0.02)} \\
\multicolumn{1}{l}{Model} &
  \multicolumn{1}{c}{OLS} &
  \multicolumn{1}{c}{IV} &
  \multicolumn{1}{c}{OLS} &
  \multicolumn{1}{c}{IV} &
  \multicolumn{1}{c}{OLS} &
  \multicolumn{1}{c}{IV} \\
\multicolumn{1}{l}{Adjusted R-squared} &
  \multicolumn{1}{c}{0.13} &
  \multicolumn{1}{c}{0.11} &
  \multicolumn{1}{c}{0.15} &
  \multicolumn{1}{c}{0.15} &
  \multicolumn{1}{c}{0.25} &
  \multicolumn{1}{c}{0.24} \\
\multicolumn{1}{l}{Number of observations} &
  \multicolumn{1}{c}{2987} &
  \multicolumn{1}{c}{2987} &
  \multicolumn{1}{c}{2564} &
  \multicolumn{1}{c}{2564} &
  \multicolumn{1}{c}{1950} &
  \multicolumn{1}{c}{1950} \\
\cline{1-7}
\end{tabular}
}
	\begin{tablenotes}
		\item {\tiny Heteroskedastic-robust standard errors are reported in parentheses. \\ $^{***}p<0.01$; $^{**}p<0.05$; $^{*}p<0.10$}
	\end{tablenotes} 
	\end{threeparttable}
\end{table}
\end{adjustwidth}
\end{frame}

\begin{frame}
\frametitle{Cognitive Test Scores} 
\begin{adjustwidth}{-2em}{-2em}
\begin{table}
	\caption{Kindergarten's Effects on Cognitive Test Scores, Standardized by Age}
	\begin{threeparttable}
		{\tiny\begin{tabular}{lllllll}
\cline{1-7}
\multicolumn{1}{c}{} &
  \multicolumn{2}{c}{2000} &
  \multicolumn{2}{c}{2007} &
  \multicolumn{2}{c}{2014}  \\
\cline{1-7}
\multicolumn{1}{l}{Kindergarten} &
  \multicolumn{1}{c}{0.06***} &
  \multicolumn{1}{c}{-0.12 } &
  \multicolumn{1}{c}{0.05** } &
  \multicolumn{1}{c}{-0.03 } &
  \multicolumn{1}{c}{0.02 } &
  \multicolumn{1}{c}{-0.05 } \\
\multicolumn{1}{l}{} &
  \multicolumn{1}{c}{(0.02)} &
  \multicolumn{1}{c}{(0.11)} &
  \multicolumn{1}{c}{(0.02)} &
  \multicolumn{1}{c}{(0.11)} &
  \multicolumn{1}{c}{(0.02)} &
  \multicolumn{1}{c}{(0.11)} \\
\multicolumn{1}{l}{Mom's yrs of education} &
  \multicolumn{1}{c}{0.01** } &
  \multicolumn{1}{c}{0.01***} &
  \multicolumn{1}{c}{0.01***} &
  \multicolumn{1}{c}{0.01** } &
  \multicolumn{1}{c}{0.02***} &
  \multicolumn{1}{c}{0.02***} \\
\multicolumn{1}{l}{} &
  \multicolumn{1}{c}{(0.00)} &
  \multicolumn{1}{c}{(0.00)} &
  \multicolumn{1}{c}{(0.00)} &
  \multicolumn{1}{c}{(0.00)} &
  \multicolumn{1}{c}{(0.00)} &
  \multicolumn{1}{c}{(0.00)} \\
\multicolumn{1}{l}{HH per-capita expenditure (1997)} &
  \multicolumn{1}{c}{0.00 } &
  \multicolumn{1}{c}{0.01 } &
  \multicolumn{1}{c}{0.01 } &
  \multicolumn{1}{c}{0.01 } &
  \multicolumn{1}{c}{0.02 } &
  \multicolumn{1}{c}{0.02 } \\
\multicolumn{1}{l}{} &
  \multicolumn{1}{c}{(0.01)} &
  \multicolumn{1}{c}{(0.01)} &
  \multicolumn{1}{c}{(0.01)} &
  \multicolumn{1}{c}{(0.01)} &
  \multicolumn{1}{c}{(0.02)} &
  \multicolumn{1}{c}{(0.02)} \\
\multicolumn{1}{l}{HH per-capita expenditure (2000)} &
  \multicolumn{1}{c}{0.01 } &
  \multicolumn{1}{c}{0.01 } &
  \multicolumn{1}{c}{0.01 } &
  \multicolumn{1}{c}{0.01 } &
  \multicolumn{1}{c}{-0.01 } &
  \multicolumn{1}{c}{-0.01 } \\
\multicolumn{1}{l}{} &
  \multicolumn{1}{c}{(0.02)} &
  \multicolumn{1}{c}{(0.02)} &
  \multicolumn{1}{c}{(0.02)} &
  \multicolumn{1}{c}{(0.02)} &
  \multicolumn{1}{c}{(0.02)} &
  \multicolumn{1}{c}{(0.02)} \\
\multicolumn{1}{l}{HH per-capita expenditure (2007)} &
  \multicolumn{1}{c}{0.04***} &
  \multicolumn{1}{c}{0.05***} &
  \multicolumn{1}{c}{0.04***} &
  \multicolumn{1}{c}{0.04***} &
  \multicolumn{1}{c}{0.05***} &
  \multicolumn{1}{c}{0.05***} \\
\multicolumn{1}{l}{} &
  \multicolumn{1}{c}{(0.01)} &
  \multicolumn{1}{c}{(0.01)} &
  \multicolumn{1}{c}{(0.01)} &
  \multicolumn{1}{c}{(0.01)} &
  \multicolumn{1}{c}{(0.02)} &
  \multicolumn{1}{c}{(0.02)} \\
\multicolumn{1}{l}{Model} &
  \multicolumn{1}{c}{OLS} &
  \multicolumn{1}{c}{IV} &
  \multicolumn{1}{c}{OLS} &
  \multicolumn{1}{c}{IV} &
  \multicolumn{1}{c}{OLS} &
  \multicolumn{1}{c}{IV} \\
\multicolumn{1}{l}{Adjusted R-squared} &
  \multicolumn{1}{c}{0.20} &
  \multicolumn{1}{c}{0.17} &
  \multicolumn{1}{c}{0.15} &
  \multicolumn{1}{c}{0.15} &
  \multicolumn{1}{c}{0.14} &
  \multicolumn{1}{c}{0.14} \\
\multicolumn{1}{l}{Number of observations} &
  \multicolumn{1}{c}{2117} &
  \multicolumn{1}{c}{2117} &
  \multicolumn{1}{c}{2117} &
  \multicolumn{1}{c}{2117} &
  \multicolumn{1}{c}{2117} &
  \multicolumn{1}{c}{2117} \\
\cline{1-7}
\end{tabular}
}
	\begin{tablenotes}
	\item {\tiny Heteroskedastic-robust standard errors are reported in parentheses. \\ $^{***}p<0.01$; $^{**}p<0.05$; $^{*}p<0.10$}
	\end{tablenotes} 
	\end{threeparttable}
\end{table}
\end{adjustwidth}
\end{frame}

\begin{frame}
\frametitle{Interpretation of Results}
\begin{itemize}
	\item Kindergarten has a positive association with schooling, mixed results for cognitive performance
	\vspace{0.1in}
	\item Results suggest `fade-out' in effect:
	\begin{itemize}
		\item For attendance and stay-on: kindergarten's effect peaked in junior high school 
	\end{itemize}
	\vspace{0.1in}
	\item IV found stronger coefficients for schooling than OLS 
	\vspace{0.1in}
	\item Opposite was true for cognitive performance
\end{itemize}
\end{frame}

\begin{frame}
\frametitle{Conclusion and Future Work}
\begin{itemize}
	\item Divergence between schooling and learning
	\begin{itemize}
		\item Mirrors broader concerns about Indonesia's education system
	\end{itemize}
	\vspace{0.1in}
	\item Could this be related to quality of kindergarten? Cost of private kindergarten? 
	\vspace{0.1in}
	\item Motivate closer look at kindergarten before public expansion is pursued:
	\begin{itemize}
		\item Kindergarten's relationship with earnings 
		\item Role of quality in Indonesian education system
		\item Effect of informal playgroups vs. formal kindergarten
	\end{itemize}
\end{itemize}
\end{frame}

\begin{frame}
\frametitle{Acknowledgements} 
I want to thank:
\vspace{0.1in}
\begin{itemize} 
	\item Professor Ranjan Shrestha for his ceaseless support, mentorship, and advice
	\item Pak Sudarno Sumarto, Pak Elan Satriawan and all of TNP2K for their gracious hospitality this summer in Jakarta, Indonesia
	\item Dan Cristol, Kim Van Deusen, and the 1693 Scholarship Program for the funds to travel to Indonesia this past summer 
\end{itemize}
\end{frame}

\end{document}